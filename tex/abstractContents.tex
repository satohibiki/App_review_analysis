%ここに論文要旨を書く. 上寄せなので注意

モバイルアプリのユーザレビューには, そのアプリに関する欠陥の報告やアプリに対する要望など開発者にとって有用な情報が多く存在する. また, Twitterのツイートにはアプリの機能などに関する情報が記述されることが先行研究で示されている. しかし, レビューやツイートの数は膨大であり, 人の手で全て確認するのは不可能に近い. 
先行研究の多くはトピックやカテゴリを事前に定義した上で分類を行うため, ``欠陥の発生している機能ごと''や``アプリに要望している内容ごと''といった粒度の細かい分類はできない. また, アプリごとのレビューの特徴に応じてトピックやカテゴリを変更することはできない. 

本研究では, レビューの中から欠陥の報告やアプリに対する要望といった開発に有用な情報を自動抽出し, 抽出した情報を元にした粒度の細かいクラスタリング手法を提案する. また, その結果を時系列やアプリごとにWebブラウザ上に出力し, 可視化する開発者支援ツールを提案する.
使用するデータは, 過去の論文で収集されたGoogle Playストアのレビュー7,912件とTwitterのツイート1,525,211件に加え, 本研究で新たに収集したGoogle Playストアのレビュー3,967件とTwitterのツイート29,360件である. このデータの中に含まれる開発に有用な情報のみを抽出し, 抽出した情報を活用することで粒度の細かいクラスタリングを可能にした. 
そして, クラスタリング結果をwebブラウザ上で可視化することで, 開発者がレビューを閲覧, 分析しやすくするためのツールを実装した. 

レビューの自動抽出の性能を評価した結果, レビューやツイートの中に開発に有用な情報があるかどうかを判断する精度は高いことが示された. 
しかし, レビューやツイートの中にある有用な情報だけを正確に抽出する精度に関しては, ある程度の高さが示されたものの, 改善の余地があることが確認された. 
クラスタリング性能に関しては, 他の手法と比較した結果, 本手法の精度の高さが確認された. 
そして, 本論文で実装した可視化ツールが開発者にとって有用であることが確認できた. 
