%ここに論文要旨を書く. 上寄せなので注意

モバイルアプリのレビューには, そのアプリに関する欠陥の報告やアプリに対する要望など開発者にとって有用な情報が多く存在する. また, 先行研究ではTwitterのツイートにはアプリに関する開発に有用な情報が記述されることが示されている. しかし, レビューやツイートの数は膨大であり, 人の手で全て確認するのは不可能に近い. 
また, レビューやツイートは構造化されていないため, 先行研究では分類の精度が上がりにくく, ``欠陥の発生している機能ごと''や``アプリに要望している内容ごと''といった粒度の細かい分類はできない. 

本研究では, レビューの中から欠陥の報告やアプリに対する要望などの開発に有用な情報を自動抽出し, 抽出した情報を元にした粒度の細かいクラスタリングを行う手法を提案する. また, その結果を時系列やアプリごとにWebブラウザ上に出力し, 可視化する開発者支援ツールを提案する.
使用するデータは, 過去の論文で収集されたGoogle Playストアのレビュー7,912件とTwitterのツイート1,525,211件に加え, 本研究で新たに収集したGoogle Playストアのレビュー3,967件とTwitterのツイート29,360件である. このデータの中に含まれる開発に有用な情報を記述している部分を抽出し, 粒度の細かいクラスタリングを可能にする手法を実装した. 
そして, クラスタリング結果をwebサイトにて可視化できる開発者支援ツールを実装した. 

性能を評価した結果, レビューの自動抽出において, レビューやツイートの中に開発に有用な情報があるかどうかを判断する精度は高いことが示された. 
しかし, レビューやツイートの中にある有用な情報だけを抽出する精度に関しては, ある程度の高さが示されたものの, 改善の余地があることが確認された. 
クラスタリング性能に関しては, 他の手法と比較した結果, 本手法の精度の高さが確認された. 
そして, 開発者にとって本論文で実装した可視化ツールが有用であることが確認できた. 
