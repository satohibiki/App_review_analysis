%ここに論文要旨を書く. 上寄せなので注意

モバイルアプリのレビューにはそのアプリに関するバグの報告や新しい機能の要望など開発者にとって有用な情報が多く存在する. しかし, レビューの数は膨大であり, 人の手で全て確認するのは不可能に近い. 近年, トピック分類やキーフレーズ抽出などレビューをマイニングする技術が注目を集めているが, レビュー文は構造化されていないため分類の精度が上がりにくく, また, 細かい機能などの粒度の高い分類はできない. 

本研究では, レビューの中にある問題のある機能やアプリに対する要望に関する報告を自動抽出してからその抽出した情報を元にクラスタリングすることによりクラスタリングの精度を向上させつつ, 粒度の高いクラスタリングを行う. また, その結果を時系列やアプリごとにWebブラウザ上に出力し, 開発者を支援する可視化ツールを提案する.
過去の論文で集められた同一期間に投稿されたGoogle Playストアのレビュー7912件とTwitterのツイート1525211件に加え, 本研究で新たに収集したGoogle Playストアのレビュー8000件とTwitterのツイート30000件を機械学習を用いて, 有用な情報を記述している部分を抽出し, クラスタリング, webサイトにて可視化を行う. 評価項目は以下の通りである.
\begin{itemize}
    \item RQ1: レビューの抽出性能の調査
    \item RQ2: 抽出文を用いたクラスタリングの性能調査
    \item RQ3: 可視化ツールの有用性
\end{itemize}

調査の結果, レビューの自動抽出によりクラスタリング性能の向上と粒度の高さが確認された. また可視化ツールの有用性を示すことができた. 
