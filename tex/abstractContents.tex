%ここに論文要旨を書く. 上寄せなので注意

モバイルアプリのレビューには, そのアプリに関する欠陥の報告やアプリに対する要望など開発者にとって有用な情報が多く存在する. また, 先行研究ではTwitterのツイートにはアプリに関する開発に有用な情報が記述されることが示されている. しかし, レビューやツイートの数は膨大であり, 人の手で全て確認するのは不可能に近い. 
また, レビュー文やツイートは構造化されていないため, 先行研究では分類の精度が上がりにくく, ``欠陥の発生している機能ごと''や``アプリに要望している内容ごと''といった粒度の高い分類はできない. 

本研究では, レビューの中から欠陥の報告やアプリに対する要望などの開発に有用な情報を自動抽出し, 抽出した情報を元にした粒度の高いクラスタリングを行う手法を提案する. また, その結果を時系列やアプリごとにWebブラウザ上に出力し, 可視化する開発者支援ツールを提案する.
過去の論文で集められたものと同一の期間に投稿されたGoogle Playストアのレビュー7,912件とTwitterのツイート1,525,211件に加え, 本研究で新たに収集したGoogle Playストアのレビュー3,967件とTwitterのツイート29,360件を機械学習を用いて, レビュー文とツイートの中に含まれる有用な情報を記述している部分を抽出し, クラスタリング, webサイトにて可視化を行う. 評価項目は以下の通りである.
\begin{itemize}
    \item RQ1: レビューの抽出性能はどの程度か
    \item RQ2: クラスタリングの性能はどの程度か
    \item RQ3: 可視化ツールの有用性はどうか
\end{itemize}

調査の結果, レビューの自動抽出の性能の高さやクラスタリング性能の高さが確認された. また可視化ツールの有用性を示すことができた. 
