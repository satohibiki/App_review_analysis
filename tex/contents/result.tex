\chapter{結論}
\label{chap:keturon}

 本論文ではレビューやツイートに含まれるキーフレーズの自動抽出および粒度の細かいクラスタリング手法を提案する. そして, マイニングした結果をwebブラウザ上で表示することで開発者を支援する可視化ツールを提案する. 
具体的には, レビューやツイートの中にあるアプリの欠陥やアプリに対する要望に関するキーフレーズを自動抽出してからその抽出したキーフレーズを意味に応じてクラスタリングする.  そして, マイニングして得られた結果を時系列やアプリごとにwebブラウザ上に出力する.
Google PlayストアだけでなくTwitterのデータも取得, 分析することによりアプリに対するユーザの幅広い情報を活用することができる. また, 抽出したキーフレーズを活用することで機能や内容などに応じた粒度の細かいクラスタリングをすることができる. 最後に得られた結果をグラフなどを用いてwebブラウザ上に表示することにより開発者がレビューを分析する効率を上げることができる. 

3つのRQを通して, 本研究で作成した自動抽出モデルはレビューの中にキーフレーズがあるかどうかを判断する精度は高いものの, レビューから正確にキーフレーズを抽出する精度に関しては改善の余地が見られた. また, 他の手法と比較することで本研究のクラスタリング手法の精度の高さが確認できた. そして, 可視化ツールの有用性を示すことができた. 

今後の課題として, 次を挙げる. 

\begin{itemize}
    \item 対象アプリを増やすことで提案手法の一般性の高さを確認する
    \item App Storeに投稿されるレビューを分析対象に含めることでデータ数を増やす
    \item モデルのパラメータやデータセットの数を変更したときに抽出の精度がどのように変化するのかを検証する
    \item トレーニングデータの精度を改善することで抽出精度を向上させる. 自動抽出の精度を向上させることでクラスタリングの精度向上につながる
    \item webブラウザへの可視化方法を工夫し, 開発者がよりレビューを分析しやすいようにデザインの改善や機能の拡張を行う
\end{itemize}
