\chapter{結論}
\label{chap:keturon}

本論文では開発者を支援する可視化ツールを提案する. 具体的には, レビューやツイートの中にあるアプリの欠陥やアプリに対する要望に関する情報を自動抽出してからその抽出した情報を元にクラスタリング,  その結果を時系列やアプリごとにWebブラウザ上に出力し, 開発者を支援するツールとなっている.
GooglePlayストアだけでなくTwitterのデータも取得, 分析することによりアプリに対するユーザの幅広い情報を活用することができる. また, 抽出したオブジェクトを活用することで細分化したクラスタリングをすることができる. 最後に抽出, クラスタリング結果をグラフなどを用いてwebブラウザ上に表示することにより開発者がレビューを理解しやすいような設計となっている. 
そして, 3つのRQにて自動抽出とクラスタリングに関する精度の高さやアンケート調査により可視化ツールの有用性を示すことができた. 

今後の課題を以下に述べる

\begin{itemize}
    \item 対象アプリを増やすことで本手法の一般性を示すとともにカテゴリーごとの結果の特徴を比較する. 
    \item トレーニングデータの精度を改善することで抽出精度を向上させる. 自動抽出の精度を向上させることでクラスタリングの精度向上につながる. 
    \item webブラウザへの可視化方法を工夫し, 開発者がよりレビューを理解しやすいように設計やデザインを改善する. 
\end{itemize}
