\chapter{序論}
\label{chap:jyoron}

%ーーーーーーーーーーーーーーーーーーーーーーーーーーーー
\section{背景}

\subsection{Google Playストア}
Google Playストア\cite{google-play-store}とは, Googleが提供しているAndroidやChromeOS向けのデジタルコンテンツ配信サービスである. Google Playストアを使用することにより, アプリやゲームの検索やインストール, 映画などのレンタルや漫画や書籍の購入などが可能となっている. 
Android向けのアプリケーションストアであった``Android Market''が2008年から始まり, 2012年3月6日に``Google Playストア''と改名された. 2022年3月をもって10周年となっており, 現在は数百万以上のコンテンツを配信している\cite{about-google-play}. 
2022年5月時点で190カ国以上の25億人のユーザーが毎月GooglePlayを使用しており, 収益は1,200億ドルに上る\cite{purnima-kochikar}. 

\subsection{X}
X(旧Twitter)\cite{twitter}とは, アメリカのX社が運営しているSNSサービスであり, 2023年7月24日に前身のTwitterから名称を変更した. 
このサービスの主な機能は「フォロー」, 「ポスト(旧ツイート)」, 「リポスト(旧リツイート)」の3つである. 相手のユーザーをフォローすることにより相手の投稿を受け取ることができるようになる. ポスト(旧ツイート)とは, 自身の書き込みを投稿することであり, 2022年時点では1日に5億件以上がポストされている\cite{aboutx}. また, 他人のポストをリポスト(旧リツイート)することにより自分のフォロワーに共有することができる. 
ポストする文章には全角で140文字, 半角で280文字の制限があるものの, 有料の``Twitter Blue''に加入することによって, 全角で2000文字, 半角で4000文字までの文章をポストできるようになる. 

\subsection{アプリのユーザーレビュー}
アプリのユーザレビュー(以下 : レビュー)とはユーザがそのアプリをインストールして実際に使用した上で, そのアプリに対する評価やコメントをする機能のことである. 
このアプリのレビューにはユーザがアプリに対して抱いている不満やアプリへの賞賛が書かれる. また, そのアプリのバグの報告や新しい機能の要望などを記述されることもある. 
アプリの開発者はレビューを参考にしてバグの修正や新しい機能の追加などアプリの品質向上や保守に努めている. 

%ーーーーーーーーーーーーーーーーーーーーーーーーーーーー

\section{本研究の目的}

モバイルアプリのユーザレビューには, バグの報告や新しい機能の追加要望など開発者にとって有用な情報が多く存在する. しかし, そのレビューの数は膨大であり, 有用なレビューが埋もれてしまうことがある. 
したがって本研究では, GooglePlayストアとXに存在するユーザレビューに含まれる有用な部分を自動で抽出し, その抽出結果をもとにクラスタリング, さらにwebブラウザ上で可視化することにより開発者をサポートする手法およびツールを提案する. 

%ーーーーーーーーーーーーーーーーーーーーーーーーーーーー

\section{本論文の構成}
本論文の構成を以下に示す.
\begin{description}

\item[第\ref{chap:kanrenkenkyuu}章 関連知識・関連研究]\mbox{}\\
本研究の関連研究であるアプリレビューのマイニング, ツイートとアプリレビューの関連性について述べる. \\

\item[第\ref{chap:kibangijyutu}章 基盤技術]\mbox{}\\
本研究で用いる基盤技術である機械学習, 自然言語処理, BERT, Chinese Whispersについて述べる. \\

\item[第\ref{chap:zisso}章 提案]\mbox{}\\
本研究で提案しているスクレイピングから可視化までの一連の手法を説明する. 特にレビューに含まれる有用な箇所の自動抽出手法, 抽出結果をもとにクラスタリングする手法, webブラウザにて出力する手法の3つの手法についてそれぞれ述べる. \\

\item[第\ref{chap:zisso}章 実装]\mbox{}\\
本研究で提案しているスクレイピングから可視化までの一連の手法を説明する. 特にレビューに含まれる有用な箇所の自動抽出手法, 抽出結果をもとにクラスタリングする手法, webブラウザにて出力する手法の3つの手法についてそれぞれ述べる. \\

\item[第\ref{chap:kekkahyouka}章 評価]\mbox{}\\
本論文で実装した手法の評価実験とその結果について述べる. 3つのResearch Questionをもとに結果を分析し, 考察する. 最後に妥当性の脅威について述べる. \\

\item[第\ref{chap:keturon}章 結論]\mbox{}\\
本研究の結論と研究を行う中で発見された今後の課題について述べる. \\

\end{description}