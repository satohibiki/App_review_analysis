\chapter{序論}
\label{chap:jyoron}

%ーーーーーーーーーーーーーーーーーーーーーーーーーーーー
\section{背景}

\subsection{Google Playストア}
Google Playストア\cite{google-play-store}とは, Googleが提供しているAndroidやChromeOS向けのデジタルコンテンツ配信サービスである. Google Playストアを使用することにより, アプリやゲームの検索やインストールが可能である. また, 映画や漫画, 書籍の購入やレンタルも可能である. 
Android向けのアプリケーションストアである``Android Market''が2008年に誕生し, 2012年3月6日に``Google Playストア''と改名した. 2022年3月をもって10周年となっており, 現在は数百万以上のコンテンツを配信している\cite{about-google-play}. 
2022年5月時点で190カ国以上の25億人のユーザーが毎月Google Playを使用しており, 収益は1,200億ドルに上る\cite{purnima-kochikar}. 

\subsection{X}
X (旧Twitter) \cite{twitter}とは, アメリカのX社が運営しているSNSサービスであり, 2023年7月24日にTwitterから名称が変更されて誕生した. 
このサービスの主な機能は``フォロー'', ``ポスト (旧ツイート) '', ``リポスト (旧リツイート) ''の3つである. 相手のユーザーをフォローすることにより相手の投稿を受け取ることができるようになる. ポストとは, 自身の書き込みを投稿することであり, 2022年時点では1日に5億件以上がポストされている\cite{aboutx}. また, 他人のポストをリポストすることにより自分のフォロワーに共有することができる. 
ポストする文章は基本的に全角で140文字, 半角で280文字の制限がある. しかし, 有料の``Twitter Blue''に加入することによって, 全角で2,000文字, 半角で4,000文字までの文章をポストできるようになる. 

アプリの名称がTwitterからXに変わったことに伴い各機能の名称が変更されたため, 本論文では, 各表記を以下の表\ref{tb:twitter}に示すように統一する. 名称がTwitterからXに変更されてからあまり時間が経過していないため, 一般的に馴染みのある旧名称を使用している. 

\begin{table}[H]
  \caption{本論文での表記}
  \label{tb:twitter}
  \begin{center}
  \begin{tabularx}{\linewidth}{X|X|X}
    \hline
    旧名称&新名称&本論文の表記\\\hline\hline
    Twitter&X&Twitter\\\hline
    ツイート&ポスト&ツイート\\\hline
    リツイート&リポスト&リツイート\\\hline
  \end{tabularx}\end{center}
\end{table}

\subsection{アプリのユーザーレビュー}
アプリのユーザーレビュー (以下 : レビュー) とはユーザーがそのアプリをインストールして実際に使用した上で, そのアプリに対する評価やコメントをする機能のことである. 
レビューにはユーザーがアプリに対して抱いている不満やアプリへの賞賛が書かれる. また, そのアプリの欠陥の報告や新しい機能の要望などが記述されることもある. 
アプリの開発者はレビューを参考にしてバグの修正や新しい機能の追加などアプリの品質向上や保守に努めている. 

%ーーーーーーーーーーーーーーーーーーーーーーーーーーーー

\section{本研究の目的}

モバイルアプリのユーザーレビューやツイートには, アプリに関する欠陥の報告やアプリに対する要望など開発者にとって有用な情報が多く存在する. しかし, レビューやツイートの数は膨大であり, 開発に有用な情報が埋もれてしまうことがある. 
したがって本研究では, Google PlayストアとTwitterに存在するレビュー, ツイートに含まれる開発に有用な情報を自動で抽出し, 抽出した情報をもとにクラスタリングする手法を提案する. さらに, webブラウザ上で表示し可視化することにより開発者を支援するツールを提案する. 

%ーーーーーーーーーーーーーーーーーーーーーーーーーーーー

\section{本論文の構成}
本論文の構成を以下に示す.
\begin{description}

\item[第\ref{chap:kanrenkenkyuu}章 関連研究]\mbox{}\\
本研究の関連研究であるレビューのマイニング, ツイートとレビューの関連性について述べる. \\

\item[第\ref{chap:kibangijyutu}章 基盤技術]\mbox{}\\
本研究で用いる基盤技術である機械学習, 自然言語処理, BERT, Chinese Whispersについて述べる. \\

\item[第\ref{chap:teian}章 提案]\mbox{}\\
本研究で提案している自動抽出とクラスタリングの手法および可視化ツールについて述べる. \\

\item[第\ref{chap:zisso}章 実装]\mbox{}\\
第\ref{chap:teian}章で提案している手法やツールの実装について述べる. 特にレビューに含まれる有用な箇所の自動抽出や抽出結果を元にしたクラスタリング, webブラウザにて可視化するツールの実装について具体的に述べる. \\

\item[第\ref{chap:kekkahyouka}章 評価]\mbox{}\\
第\ref{chap:zisso}章で実装した手法およびツールの評価実験とその結果について述べる. 3つのResearch Questionをもとに結果を分析し, 考察する. 最後に妥当性の脅威について述べる. \\

\item[第\ref{chap:keturon}章 結論]\mbox{}\\
本研究の結論と研究を行う中で発見された今後の課題について述べる. \\

\end{description}