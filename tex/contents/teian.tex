\chapter{提案}
\label{chap:teian}

%ーーーーーーーーーーーーーーーーーーーーーーーーーーーー

\section{概要}
本研究で提案している手法やツールの流れを下記に示す. 

\begin{enumerate}
  \item GooglePlayとTwitterに投稿されたレビューをスクレイピングして取得, 前処理
  \item レビュー文に含まれる欠陥の報告やアプリに対する要望を示す箇所を自動抽出
  \item 抽出結果を利用しクラスタリング
  \item 分析された結果をwebブラウザ上に可視化
\end{enumerate}

提案手法の流れを図\ref{fig:nagare}に示す. 

\begin{figure}[hbtp]
  \centering
  \includegraphics[width=\linewidth]
       {contents/images/zisso_nagare.png}
  \caption{実装した提案手法の流れ\label{fig:nagare}}
\end{figure}

%ーーーーーーーーーーーーーーーーーーーーーーーーーーーー

\section{対象アプリとレビュー}
本研究では川面による先行研究\cite{kawatsura}のデータセットを使用するため対象アプリは先行研究のアプリに合わせている. 注意点として, BuzzVideoは2022年3月をもってサービスを終了しているため, 本研究で収集の対象となるアプリには含まれない. そのため, 本研究で新たに取得するレビューの対象となるアプリはBuzzVideo以外の12のアプリである. 
対象となっているアプリを表\ref{tb:taisyouapuri}に示す. 
\begin{table}[htbp]
  \caption{本研究の対象アプリ一覧}
  \label{tb:taisyouapuri}
  \begin{center}
  \begin{tabularx}{\linewidth}{X|l|X}
    \hline
    \mbox{アプリ名}\mbox{(一部略称)}&\mbox{Google Playストアの}\mbox{パッケージID}&\mbox{Twitterの}\mbox{検索キーワード}\\\hline\hline
    にゃんトーク&com.akvelon.meowtalk&にゃんトーク\\\hline
    スマートニュース&jp.gocro.smartnews.android&スマートニュース\\\hline
    PayPay&jp.ne.paypay.android.app&paypay\\\hline
    Coke ON&com.coke.cokeon&coke on\\\hline
    Google Fit&com.google.android.apps.fitness&google fit\\\hline
    Simeji&com.adamrocker.android.input.simeji&simeji\\\hline
    Lemon8&com.bd.nproject&lemon8\\\hline
    楽天ペイ&jp.co.rakuten.pay&楽天ペイ\\\hline
    majica&com.donki.majica&majica\\\hline
    LINE MUSIC&jp.linecorp.linemusic.android&line music\\\hline
    BuzzVideo&com.ss.android.article.topbuzzvideo&buzzvideo\\\hline
    ファミペイ&jp.co.family.familymart\verb|_|app&ファミペイ\\\hline
    CapCut&com.lemon.lvoverseas&capcut\\\hline
  \end{tabularx}\end{center}
\end{table}

%ーーーーーーーーーーーーーーーーーーーーーーーーーーーー

\section{事前準備}
本研究で対象としているアプリに関するGooglePlayストアとTwitterのレビュー及びツイートをスクレイピングし, 分析対象となるデータセットを作成する. 本研究にて使用するデータセットは先行研究によって作成された13個のアプリレビューのデータセットに加え, 本研究で新たに収集した12個のアプリレビューのデータセットを使用する. 
対象アプリを先行研究に合わせた理由としては, 現在, Twitterの利用規約によりツイートの収集可能な数に制限がある影響で, 本研究だけでは十分な数のツイートが収集できなかったためである. Twitterのツイート取得数に関する制限は\ref{sec:x}で詳しく述べる. 
そして, スクレイピングしたデータから有用な箇所を自動抽出するために一般的な自然言語処理で行われる前処理を行う. 

%ーーーーーーーーーーーーーーーーーーーーーーーーーーーー

\section{有用な箇所の自動抽出}
\subsection{概要}
本研究における自動抽出の役割は以下の2つである. 
\begin{itemize}
  \item 前処理された大量のレビューやツイートから開発に有用な文章のみを絞り込む
  \item 文章中からバグの報告やアプリに対する要望に関して記述している部分(以下 : オブジェクト)を抽出する
\end{itemize}
これにより開発に有用な情報のみが取得できる. 

\subsection{使用するモデル}
日本語のデータで事前学習済みの言語表現モデルである日本語BERTに対して質問応答形式のfune-tuningを行うことで自動抽出器を生成する. 
抽出対象となる元の文章の特徴をモデルに理解させるために質問文にその文章がGoogle PlayストアのレビューなのかTwitterのツイートなのかという「カテゴリー」の情報と「アプリ名」という2つの情報を加えることにより学習性能を上げる. 
図\ref{fig:fine-tuning}に質問応答形式によるfine-tuningのモデルを示す. 

\begin{figure}[hbtp]
  \centering
  \includegraphics[scale=0.3]
       {contents/images/fine-tuning.png}
  \caption{ファインチューニング\label{fig:fine-tuning}}
\end{figure}

また, 図\ref{fig:answer}に質問文とその答えの例を示す. 質問文にアプリの欠陥やアプリに対する要望を尋ねる文章を与え, その答えとしてオブジェクトを返すようにしている. 

\begin{figure}[hbtp]
  \centering
  \includegraphics[scale=0.4]
       {contents/images/answer.png}
  \caption{オブジェクトの抽出例\label{fig:answer}}
\end{figure}

%ーーーーーーーーーーーーーーーーーーーーーーーーーーーー

\section{クラスタリング}
\subsection{概要}
本研究におけるクラスタリングの役割は以下の2つである. 
\begin{itemize}
  \item 抽出された各オブジェクトを類似度に応じてクラスタリングする
  \item それぞれのクラスタにおける名称を決める
\end{itemize}

\subsection{クラスタリング手法}
抽出したオブジェクトのクラスタリングには先行研究のグラフクラスタリング手法\cite{sira}を参考にして実装する. 既存研究では英語の文章をベクトルに変換するためにUniversal Sentence Encoder (USE)を使用しているが, 今回は対象が日本語の文章であるため, Sentence-BERTの日本語モデルを使用する. Sentence-BERT\cite{sentence-bert}とは, 事前学習されたBERTモデルとSiamese Networkを使い, 高品質な文ベクトルを作る手法である. このモデルを使用することで, 高品質な文ベクトルが作成できる. 
したがって本研究でのクラスタリング手法は下記の手順ある. 
% \begin{enumerate}
%   \item Universal Sentence Encoder (USE)を用いて, 問題のある特徴語句を512次元のベクトルに変換
%   \item 重み付き無向グラフを構成し, 各問題機能をノードとし, 2つの問題機能のUSEベクトル間のコサイン類似度スコアをノード間の重みとする. スコアがある閾値以上の場合, 2つのノード間にエッジを追加. この閾値は入力ハイパーパラメータであり, 問題のある機能間の意味的相関を測定するために使用される. 閾値が高いほどクラスタの結束力が高まる. 
%   \item このグラフに対して, Chinese Whispers (CW)を実行し, 問題のある特徴量をクラスタリング
% \end{enumerate}
\begin{enumerate}
  \item Sentence-BERTの日本語モデルを用いて, 抽出した文章をベクトルに変換
  \item 重み付き無向グラフを構成し, 抽出したオブジェクトをノードとし, 2つの抽出した文章間のベクトル間のコサイン類似度スコアをノード間の重みとする. スコアがある閾値以上の場合, 2つのノード間にエッジを追加. この閾値は入力ハイパーパラメータであり, 抽出されたオブジェクト間の意味的相関を測定するために使用される. 閾値が高いほどクラスタの結束力が高まる. 
  \item このグラフに対して, Chinese Whispers (CW)を実行し, 問題のある特徴量をクラスタリング
\end{enumerate}

以下の図\ref{fig:clustering}にクラスタリングの概略図を示す. 丸で示されているのが抽出されたオブジェクトを表すノードであり, 点線がエッジである. このエッジは設定された閾値によって変化する. 
\begin{figure}[hbtp]
  \centering
  \includegraphics[scale=0.4]
       {contents/images/clustering.png}
  \caption{グラフクラスタリングの概略図\label{fig:clustering}}
\end{figure}

\subsection{各クラスタの名称}
クラスタを作成したら各クラスタの特徴を表す名称を決定する. 
この名称の決定にはKeyBERT\cite{keybert}を使用したキーフレーズの抽出によって実現する. KeyBERTとは, BERTの埋め込みを活用して, 文書に最も類似したキーワードとキーフレーズを作成するキーワード抽出技法である. 
具体的な手順は以下に示す通りである. 

\begin{enumerate}
  \item 文書レベルの表現を得るために, BERTを用いて文書埋め込みを抽出する
  \item N-gramの単語やフレーズについて単語埋め込みを抽出する
  \item コサイン類似度を用いて, 文書に最も類似する単語やフレーズを見つける. 最も類似している単語は, 文書全体を最もよく表現する単語として特定される
\end{enumerate}

%ーーーーーーーーーーーーーーーーーーーーーーーーーーーー

\section{画面出力・可視化}
クラスタリング結果を用いて画面出力し, 可視化を行う. クラスタごとにまとめて表示することで開発者が各レビューを確認しやすいようにする. 
開発者はアプリの修正やアップデートを行った後でユーザがどのようなレビューを挙げているかを特に確認したい. したがって, 特定の機能や期間でのレビュー内容を確認できるよう, レビューが投稿された期間やキーワードで絞り込むことが可能な検索機能を実装した. 
そして, レビューの特徴を確認するために以下2つのグラフを作成した.
\begin{itemize}
  \item 日ごとのレビュー数を表す折れ線グラフ
  \item クラスタに含まれるレビュー数の上位10個を表す棒グラフ
\end{itemize}
この2つのグラフは検索結果に応じて動的に変化するようになっている. 以上の機能をwebアプリとして実装することにより, 開発者がレビューを理解, 分析しやすいように実装されている. 