\documentclass[report, a4paper, 12pt]{jsbook}

% しおり機能とコメント機能 (必ず, graphicx系より先に書く事)
\usepackage[dvipdfmx]{hyperref}
\usepackage{pxjahyper}

\usepackage[dvipdfmx]{color}
\usepackage[dvipdfmx]{xcolor}
\usepackage[dvipdfmx]{graphicx}
\usepackage{enumerate}
\usepackage{cover}
\usepackage{abstract}
\usepackage{thesis}
\usepackage{here}
\usepackage{url}
\usepackage[toc,page]{appendix}
\usepackage{tabularx}
\usepackage{multirow}
\usepackage{amsmath}
\usepackage{booktabs}
\usepackage{algorithmic}
\usepackage{algorithm}
\usepackage{float}

% Usage用
%\usepackage{minted}
\usepackage{listings, jlisting}
\lstset{%listings の表示設定
  breaklines = true,%自動で折り返す。
  tabsize = 2,%tabsize
  frame=shadowbox,%枠を上下左右に表示する
  basicstyle = \footnotesize\ttfamily,%\footnotesize,
  keywordstyle=\bfseries\color{green!40!black},
  commentstyle=\itshape\color{purple!40!black},
  lineskip=-0.5zw,
  showstringspaces=false,%文字列中のスペースをちゃんと" "と表示。(デフォルトは文字列中のスペースは可視設定)
  numbers=left,%行番号を左に
  framexleftmargin=6truemm,%行番号をフレーム内に
  numberstyle=\scriptsize,%行番号のサイズ
  stepnumber=1,%1行おきに行番号を
  numbersep=1zw,%ソースと行番号の間隔
  language = C
}%言語設定

\renewcommand{\thelstnumber}{\arabic{lstnumber}:}

\studentid{62008972}
\studentname{佐藤響}
\studentruby{さとうひびき}

%makecover用設定
\covername{佐\ 藤\ \ 響}
\coveryear{令\ 和\ 5\ 年\ 度}
\covertitle{Google PlayストアとXにおける\mbox{}\\ユーザレビューを用いた\mbox{}\\開発者支援ツール}

%makeabstract用設定
\absttitle{Google PlayストアとXにおける\mbox{}\\ユーザレビューを用いた開発者支援ツール}
\abstcontent{%ここに論文要旨を書く. 上寄せなので注意

モバイルアプリのレビューには, そのアプリに関する欠陥の報告やアプリに対する要望など開発者にとって有用な情報が多く存在する. また, 先行研究ではTwitterのツイートでアプリに関する開発に有用な情報が記述されることが示されている. しかし, レビューやツイートの数は膨大であり, 人の手で全て確認するのは不可能に近い. 
また, レビュー文やツイートは構造化されていないため, 先行研究では分類の精度が上がりにくく, ``欠陥の発生している機能ごと''や``アプリに要望している内容ごと''といった粒度の高い分類はできない. 

本研究では, レビューの中から欠陥の報告やアプリに対する要望などの開発に有用な情報を自動抽出, 抽出した情報を元にした粒度の高いクラスタリングを行う手法を提案する. また, その結果を時系列やアプリごとにWebブラウザ上に出力し, 可視化する開発者支援ツールを提案する.
過去の論文で集められた同一期間に投稿されたGoogle Playストアのレビュー7,912件とTwitterのツイート1,525,211件に加え, 本研究で新たに収集したGoogle Playストアのレビュー3,967件とTwitterのツイート29,360件を機械学習を用いて, レビュー文とツイートの中に含まれる有用な情報を記述している部分を抽出し, クラスタリング, webサイトにて可視化を行う. 評価項目は以下の通りである.
\begin{itemize}
    \item RQ1: レビューの抽出性能はどの程度か
    \item RQ2: クラスタリングの性能はどの程度か
    \item RQ3: 可視化ツールの有用性はどうか
\end{itemize}

調査の結果, レビューの自動抽出の性能の高さやクラスタリングの性能の高さと粒度の細かさが確認された. また可視化ツールの有用性を示すことができた. 
}

%maketitle用設定
\title{Google PlayストアとXにおける\mbox{}\\ユーザレビューを用いた開発者支援ツール}
\author{佐藤響}
\date{}

\begin{document}
\makecover    % 表紙ページ生成
\makeabstract % 論文要旨ページ生成
\maketitle    % 内表紙ページ生成
\maketoc      % 目次ページ生成

%\layout

%%%%%%%%%%%%%%%%%%%%%%%%%%%%%%%%%%%%%%%%%%%%%%%%%%%%%%%%%%%%%%%%%%
% 論文本体
% ここに論文の本体のtexファイルを \input{ファイルパス} で入れる. %
% ex. \chapter{テンプレートの使用方法}

この章では, 卒論テンプレートの構成と使用方法について述べる.

\section{設定方法}
thesis.tex を自分のパラメータに変更する.

\section{テンプレートのタイプセット}
この卒論テンプレートから卒論をタイプセットするにあたり, 以下の3つが利用可能である.
\begin{itemize}
\item make
\item omake
\item latexmk
\end{itemize}
以下でそれぞれのコマンドでのタイプセット方法について説明する.

\subsection{make}
% によるタイプセットは以下の利点, 欠点がある.
% \begin{description}
% \item [利点:] 非常に簡単にタイプセットを行える.
% \item [欠点:] makeをインストールしなければならない.
% \end{description}

makeでタイプセットを行う方法をコード\ref{code:make}で示す.
このように端末に打ち込むことで, 卒論本体である thesis.pdf が生成される.
また, 生成された中間ファイルを消去する場合は, コード\ref{code:make-clean}を利用すればよい.

\begin{minipage}{\textwidth}
\begin{lstlisting}[caption={makeコマンド}, label={code:make}, language={sh}]
make
\end{lstlisting}

\begin{lstlisting}[caption={中間ファイルの除去}, label={code:make-clean}, language={sh}]
make clean
\end{lstlisting}

\lstinputlisting[caption={Makefile}, label={code:Makefile}, language={make}]{Makefile}
\end{minipage}

\subsection{omake}
後で書く

\subsection{latexmk}
latexmkはTeXLiveに標準で入っている自動タイプセットツールである.
このツールを利用するには, コード\ref{code:latexmk}を実行すればよい. 
また, latexmkの特徴としてファイルの更新の監視を行い, 更新時に自動でタイプセットを行うことが出来る.
この機能を使用するためには, コード\ref{code:latexmk-pvc}を利用すればよい.

\begin{minipage}{\textwidth}
\begin{lstlisting}[caption={latexmkコマンド}, label={code:latexmk}, language={sh}]
latexmk thesis.tex
\end{lstlisting}

\begin{lstlisting}[caption={latexmkによる自動タイプセット}, label={code:latexmk-pvc}, language={sh}]
latexmk -pvc thesis.tex
\end{lstlisting}

\lstinputlisting[caption={.latexmkrc}, label={code:latexmkrc}, language={perl}]{.latexmkrc}
\end{minipage}                                   %
%%%%%%%%%%%%%%%%%%%%%%%%%%%%%%%%%%%%%%%%%%%%%%%%%%%%%%%%%%%%%%%%%%
\chapter{序論}
\label{chap:jyoron}

%ーーーーーーーーーーーーーーーーーーーーーーーーーーーー
\section{背景}

\subsection{Google Playストア}
Google Playストア\cite{google-play-store}とは, Googleが提供しているAndroidやChromeOS向けのデジタルコンテンツ配信サービスである. Google Playストアを使用することにより, アプリやゲームの検索やインストールが可能である. また, 映画や漫画, 書籍の購入やレンタルも可能である. 
Android向けのアプリケーションストアである``Android Market''が2008年に誕生し, 2012年3月6日に``Google Playストア''と改名した. 2022年3月をもって10周年となっており, 現在は数百万以上のコンテンツを配信している\cite{about-google-play}. 
2022年5月時点で190カ国以上の25億人のユーザーが毎月Google Playを使用しており, 収益は1,200億ドルに上る\cite{purnima-kochikar}. 

\subsection{X}
X(旧Twitter)\cite{twitter}とは, アメリカのX社が運営しているSNSサービスであり, 2023年7月24日にTwitterから名称が変更されて誕生した. 
このサービスの主な機能は「フォロー」, 「ポスト(旧ツイート)」, 「リポスト(旧リツイート)」の3つである. 相手のユーザーをフォローすることにより相手の投稿を受け取ることができるようになる. ポスト(旧ツイート)とは, 自身の書き込みを投稿することであり, 2022年時点では1日に5億件以上がポストされている\cite{aboutx}. また, 他人のポストをリポスト(旧リツイート)することにより自分のフォロワーに共有することができる. 
ポストする文章は基本的に全角で140文字, 半角で280文字の制限がある. しかし, 有料の``Twitter Blue''に加入することによって, 全角で2000文字, 半角で4000文字までの文章をポストできるようになる. 

\subsection{アプリのユーザーレビュー}
アプリのユーザレビュー(以下 : レビュー)とはユーザがそのアプリをインストールして実際に使用した上で, そのアプリに対する評価やコメントをする機能のことである. 
レビューにはユーザがアプリに対して抱いている不満やアプリへの賞賛が書かれる. また, そのアプリの欠陥の報告や新しい機能の要望などが記述されることもある. 
アプリの開発者はレビューを参考にしてバグの修正や新しい機能の追加などアプリの品質向上や保守に努めている. 

%ーーーーーーーーーーーーーーーーーーーーーーーーーーーー

\section{本研究の目的}

モバイルアプリのユーザレビューやツイートには, アプリに関する欠陥の報告やアプリに対する要望など開発者にとって有用な情報が多く存在する. しかし, レビューやツイートの数は膨大であり, 開発に有用な情報が埋もれてしまうことがある. 
したがって本研究では, Google PlayストアとXに存在するレビュー, ポストに含まれる開発に有用な情報を自動で抽出し, 抽出した情報をもとにクラスタリングする手法を提案する. さらに, webブラウザ上で表示し可視化することにより開発者を支援するツールを提案する. 

%ーーーーーーーーーーーーーーーーーーーーーーーーーーーー

\section{本論文の構成}
本論文の構成を以下に示す.
\begin{description}

\item[第\ref{chap:kanrenkenkyuu}章 関連知識・関連研究]\mbox{}\\
本研究の関連研究であるレビューのマイニング, ツイートとレビューの関連性について述べる. \\

\item[第\ref{chap:kibangijyutu}章 基盤技術]\mbox{}\\
本研究で用いる基盤技術である機械学習, 自然言語処理, BERT, Chinese Whispersについて述べる. \\

\item[第\ref{chap:teian}章 提案]\mbox{}\\
本研究で提案している自動抽出とクラスタリングの手法および可視化ツールに関して説明する. \\

\item[第\ref{chap:zisso}章 実装]\mbox{}\\
第\ref{chap:teian}章で提案している手法やツールの実装に関して説明する. 特にレビューに含まれる有用な箇所の自動抽出や抽出結果を元にしたクラスタリング, webブラウザにて可視化するツールの実装について具体的に述べる. \\

\item[第\ref{chap:kekkahyouka}章 評価]\mbox{}\\
第\ref{chap:zisso}章で実装した手法およびツールの評価実験とその結果について述べる. 4つのResearch Questionをもとに結果を分析し, 考察する. 最後に妥当性の脅威について述べる. \\

\item[第\ref{chap:keturon}章 結論]\mbox{}\\
本研究の結論と研究を行う中で発見された今後の課題について述べる. \\

\end{description}
\chapter{関連研究}
\label{chap:kanrenkenkyuu}




%ーーーーーーーーーーーーーーーーーーーーーーーーーーーー

\section{アプリレビューのマイニング}
近年, テキストをマイニングするための自動化技術 (トピック分類やキーワード抽出など)  をアプリレビューに応用する研究が進んでいる. これらの技術によって, 開発者がアプリレビューを理解・分析するために必要とする労力を軽減することに繋がっている. 
INFAR\cite{infar}はレビューから洞察を発見し, レビュー文を事前に定義されたトピックに分類したのちに要約を生成する手法である. 定義されるトピックはクラッシュやGUIなどとなっている. 
Wanら\cite{dsa}はアプリレビューの分類の精度を上げるためにフレームセマンティックを用いてアプリレビューに注釈をつけて自動分類するアプローチを提案している. フレームセマンティックとは様々な関係するフレーム要素の状況を記述する概略表現である. この自動分類ではレビューを``Bug report'', ``Feature Request'', ``Others''の3つに分類している. 
また, SUR-Miner\cite{sur-miner}はレビューを表\ref{tb:categories}に示した5つのカテゴリに分類し, 依存関係解析やPart-of-Speechパターンなどの技術を使用して, アプリレビューからいくつかの側面を抽出する. そして最後に概要を可視化する. 
これらの分類における課題として, 事前に定義されたカテゴリやトピックにしか分類することができないためクラスの数が固定されてしまうことである. また, 分類の粒度が粗いことも課題として挙げられる. 

他にも, Casper\cite{caspar}というレビューからアプリの問題に関してユーザーが報告したミニストーリー (ユーザーアクションと関連するアプリの動作という2種類のイベント)  を抽出し, 合成するための手法が提案されている. 

\begin{table}[H]
  \small
  \caption{Definition of Five Review Categories( \cite{sur-miner} p.763, Table I)}
  \label{tb:categories}
  \begin{center}
    \begin{tabularx}{\linewidth}{l|l|X}
      \hline
      Category&Definition&Examples\\\hline\hline
      Praise&
      \begin{tabular}{l}
        Expressing emotions \\without specific reasons
      \end{tabular}&
      \begin{tabular}{X}
        Excellent!\\I love it!\\Amazing!
      \end{tabular}\\\hline
      Aspect Evaluation&
      \begin{tabular}{l}
        Expressing opinions \\for specific aspects
      \end{tabular}&
      \begin{tabular}{X}
        The UI is convenient.\\I like the prediction text.
      \end{tabular}\\\hline
      Bug Report&
      \begin{tabular}{l}
        Reporting bugs, \\glitches or problems
      \end{tabular}&
      \begin{tabular}{X}
        It always force closes \\when I click the “.com” button.
      \end{tabular}\\\hline
      Feature Request&
      \begin{tabular}{l}
        Suggestions or \\new feature requests
      \end{tabular}&
      \begin{tabular}{X}
        It would be better \\if I could give opinion on it. \\It's a pity it doesn't support \\Chinese.\\I wish there was a “deny” button.
      \end{tabular}\\\hline
      Others&
      \begin{tabular}{l}
        Other categories that \\are defined in 表\ref{tb:categories}
      \end{tabular}&
      \begin{tabular}{X}
        I've been playing \\it for three years
      \end{tabular}\\\hline
    \end{tabularx}
  \end{center}
\end{table}



%ーーーーーーーーーーーーーーーーーーーーーーーーーーーー
\section{ツイートとアプリレビューの関連性}
Gouriら\cite{tweetapp}はTwitterからのユーザーフィードバックをタイミングと内容の2つの観点から評価し, App Storeのレビューと比較した. 
ツイートとアプリレビューをテキスト分析して, LDA\cite{lda}を用いて分類した. その結果, 426件のツイートと2,383件のレビュー (バグの報告と機能の要求)  のタイミング分析では, 約15\%が最初にTwitterに表示されることが示された. 
また, 15\%のツイートのうち, 72\%はモバイルアプリの機能または動作の側面に関連しているものであった. 一方で, App Storeのレビューはモバイルアプリの機能または動作の側面に関連しているものが全体の80\%であった. 
さらに, 表\ref{tb:topic}に示されているように, ツイートにはアプリに関連する重大な問題や深刻な問題を示すトピックがつぶやかれている事例を少なくとも6つ確認することができる. App Storeのトピックもタイミングや回数などの詳細が追加されているものの, 同様の情報を示している. 

\begin{table}[H]
  \caption{Topic analysis from LDA( \cite{tweetapp} p.20, Table IV)}
  \label{tb:topic}
  \begin{center}
    \begin{tabularx}{\linewidth}{|l|l|X|X|}
      \hline
      App&Topic\#&Topics on Twitter&Topics in App Store reviews\\\hline
      Dropbox&
      \begin{tabular}{l}
        1\\2
      \end{tabular}&
      \begin{tabular}{X}
        unable, file, sync, ac- \\cess, try \\connect, fix, mac, open, \\crash
      \end{tabular}&
      \begin{tabular}{X}
        file, upload, unable, \\sync, horrible, time \\crash, every, time, try, \\three
      \end{tabular}\\\hline
      Google cast&
      \begin{tabular}{l}
        1\\2
      \end{tabular}&
      \begin{tabular}{X}
        googlecast, work, bring, \\resolve, session \\reboot, router, tv, add, \\screen
      \end{tabular}&
      \begin{tabular}{X}
        problem, fine, tv, con- \\nect, work \\googlecast, sometimes, \\screen, work, win
      \end{tabular}\\\hline
      LinkedIn&
      \begin{tabular}{l}
        1\\2
      \end{tabular}&
      \begin{tabular}{X}
        wish, meet, announce, \\connect, use \\use, prospects, connect, \\download, wish
      \end{tabular}&
      \begin{tabular}{X}
        option, add, thanks, vi- \\brate, please \\say, make, account, \\launch, fight
      \end{tabular}\\\hline
    \end{tabularx}
  \end{center}
\end{table}
\chapter{基盤技術}
\label{chap:kibangijyutu}


%ーーーーーーーーーーーーーーーーーーーーーーーーーーーー

\section{概要}
キーフレーズの自動抽出で使用したモデルは自然言語処理タスクに特化したBERTである. 本研究では, BERTの事前学習済みモデルをファインチューニングすることで自動抽出モデルを生成している. 
モデルのファインチューニングには機械学習の一種である教師あり学習を用いている. 
そして, 自動抽出したキーフレーズのクラスタリングにはグラフクラスリング手法の一種であるChinese Whispersを使用している. 
次にそれぞれの技術に関して詳細を示す. 

%ーーーーーーーーーーーーーーーーーーーーーーーーーーーー

\section{BERT}
\subsection{BERTとは}\label{aboutbert}
BERT (Bidirectional Encoder Representations from Transformers) \cite{bert}とは, GoogleのJacob Devlinらによって2018年秋に提案された言語表現モデルである. BERTは, ラベル付けされていないテキストから深い双方向表現を事前学習するように設計されている. その結果, 質問応答や言語推論などの幅広いタスクのためのモデルを作成するために, 出力層を1つ追加するだけで微調整することができる. 
BERTは11の自然言語処理タスクにおいて, GLUEスコアやSQuAD v1.1による質問応答テストのF1スコアなどで向上が確認された. \cite{bert}

\subsection{学習方法}
BERTは事前学習とファインチューニングの2つのステップからなり, どちらのステップにおいてもTransformerモデルを用いる. Transformerとは深層学習のベースとなっているモデルである. RNNやCNNには並列処理ができないという欠点があるが, Transformerは再帰や畳み込みは一切行わず, Attentionのみを用いることで並列化を可能にした. 基本的な構成はmulti-head attention層, add \& norm層 (残差結合 \& layer normalization) , position-wise FNN層となっている. 

\subsection{事前学習}
事前学習では, Masked Language ModelとNext Sentence Predictionの2つのタスクを解く. Masked Language Modelは入力されたトークンをランダムにマスクし, マスクされたトークンを他のトークンから予測するタスクである. Next Sentence Predictionではある文章に対して, その後に出現する文を並べたペアを正例, ランダムな文章を並べたペアを負例として識別する問題を解く. この問題を解くことにより, 2つの文章が隣り合っているかどうかを予測するよう学習する. 
この2つのタスクは自己教師あり学習である. 自己教師あり学習とは, ラベルが付与されていない大量のデータセットを用いて, プレテキストタスク (擬似的なラベルが自動生成された代わりのタスク) を解くための学習方法である. 自己教師あり学習により人の手作業によるラベル付けを必要とせずに大量のデータで学習することができる. 

\subsection{ファインチューニング}
ファインチューニングでは, ラベル付きデータを用いて特定のタスクに特化するように学習させる. \ref{aboutbert}項で述べた通り, 解きたいタスクに応じてTransformerの上に出力層を1つ追加する. そしてラベル付きデータを用いて出力層とTransformerのパラメータを更新する. 

%ーーーーーーーーーーーーーーーーーーーーーーーーーーーー

\section{機械学習}
\subsection{機械学習とは}
機械学習とは, データを分析するための手法の1つであり, 大量のデータをコンピュータが学習し, データに潜んでいるルールやパターンを発見する手法である. コンピュータが自ら学習した成果を用いて未知のデータの予測や発見を可能としており, 反復的に学習させることでデータの中にある規則性, 特徴を発見することができる. 機械学習は現在, 生物学や自動運転, 金融工学などさまざまな分野で大きな影響を与えている. 

機械学習の学習方法には教師あり学習, 教師なし学習, 強化学習の3種類が存在する. モデルのファインチューニングにおいて必要となる教師あり学習を中心に, 各学習方法について説明する. 

\subsection{教師あり学習}
読み込んだデータから入力と出力の関係を学習させ, データ間の関係性を学習させる手法である. 学習データには事前に``正解''のラベルが付与される. 
入力された値と正解のラベルのセットを繰り返し学習させることで, 未知の入力された値に対して正解となるデータを予測し, 出力することが可能となっている. 
教師あり学習の具体例には需要予測や株価予測, 画像認識などが挙げられる. 

モデルのファインチューニングではこの教師あり学習を使用している. 元のレビューを入力, レビューから抽出するべきキーフレーズを出力として, 入力と出力の関係をモデルに学習させることにより自動抽出モデルを生成した.

\subsection{教師なし学習}
学習データに``正解''のラベルは付与せずに, データセットのパターンからデータの関係を認知させる学習手法である. 主に正解と不正解が明確でない問題の解決策として用いられる. 
% 与えられたデータを繰り返し学習することによりそのデータにどのようなパターンが存在するかをコンピュータ自身が見つけ出すことができる. 
% そのため, 未知のデータに対する予測, 識別を可能とする. 

% 教師なし学習で行う代表的な例は``クラスタリング''と``次元の削減''である. クラスタリングは複数のデータをそのデータの特徴に応じて幾つかのグループに分けることである. 次元の削減とはデータの次元数を減らすことでデータの特徴を表す情報を抽出することである. 

\subsection{強化学習}
環境と相互作用しながら報酬をもとに行動を学習する手法である. 強化学習には方策に従って行動を学習する主体である``エージェント''と状態と報酬をエージェントに返し, エージェントが行動を与える対象である``環境''の2つが存在する. 

% 強化学習の具体例には将棋や囲碁などのゲームAIやロボットの単純動作の獲得などが挙げられる. Google社のAlphaGoというAIが韓国の囲碁プロ棋士に勝ったことで大きな話題を呼んだ学習方法である. 

%ーーーーーーーーーーーーーーーーーーーーーーーーーーーー

% \section{自然言語処理}
% 自然言語処理 (Natural Language Processing) とは, 人間が使用する言語 (自然言語) をコンピュータが分析する技術である. 自然言語処理によって, 自然言語による大量のテキストデータが持つ意味を解析, 処理することができる. 
% 自然言語には言葉の曖昧性や意味の重複が含まれている. 例えば, 同じ単語でも文脈が異なると意味が異なることや同じ意味でも様々な表現があることがある. したがって, 意味解釈や文脈解析でコンピュータが人間の意図に反した処理をしてしまうことが課題とされている. 

% 自然言語処理の活用事例として質問への回答, 文の要約や翻訳, テキスト分類などが挙げられる. 


%ーーーーーーーーーーーーーーーーーーーーーーーーーーーー

\section{Chinese Whispers}
\subsection{Chinese Whispersとは}
Chinese Whispers (CW) \cite{chinese-whispers}とはグラフクラスタリングのランダム化アルゴリズムである. Chinese Whispersは, 事前にクラスタの数を指定しないため, 異なるサイズのクラスタリングを扱うことができる. そのため, クラスタの数が事前にわからないNLP問題に適している. 
Chinese Whispersは次に示す4つの手順でクラスタリングを行う. 

\begin{enumerate}[label=\textbf{\arabic*.}]
  \item \textbf{グラフの構築}\\
  ノードとエッジからなる重み付き無向グラフを作成する. これがクラスタリングの対象となるグラフである. ノードはテキスト文書などのデータの要素を表し, エッジはノード間の関連性を表す. \\
  
  \item \textbf{クラスタリングの開始}\\
  クラスタリングの開始時点では, 各ノードはそれぞれ異なるクラスタに属するものとする. すなわち, 1つのノードに対して1つのクラスタが割り振られる. \\
  
  \item \textbf{グラフの反復処理とクラスタの更新}\\
  ノードは少数の反復ステップによってグラフを処理し, 接続されたノードの情報を受け取り, 接続するノードの中で最も近いノードのクラスタを継承する. これは現在のノードに対するエッジの重みが最大となるクラスタであるため, 類似したノードが同じクラスタにグループ化される. 
  最も強いクラスタが複数ある場合はランダムで1つ選ばれる. 一方でどのエッジにも接続されていないノードはクラスタリングのプロセスから除外されるため, 一部のノードはクラスタリングされないことがある. \\
  
  \item \textbf{クラスタリングの収束}\\
  グラフの反復処理とクラスタの更新を繰り返すことにより, クラスタが再構築され, 新たなクラスタの構築が進行する. このステップをクラスタリングが収束するまで繰り返した結果, 各ノードは最終的なクラスタに所属する. 
\end{enumerate}
\chapter{提案}
\label{chap:teian}

%ーーーーーーーーーーーーーーーーーーーーーーーーーーーー

\section{概要}
本研究で提案している手法やツールの流れを下記に示す. 

\begin{enumerate}
  \item GooglePlayとTwitterに投稿されたレビューをスクレイピングして取得, 前処理
  \item レビュー文に含まれる欠陥の報告やアプリに対する要望を示す箇所を自動抽出
  \item 抽出結果を利用しクラスタリング
  \item 分析された結果をwebブラウザ上に可視化
\end{enumerate}

提案手法の流れを図\ref{fig:nagare}に示す. 

\begin{figure}[hbtp]
  \centering
  \includegraphics[width=\linewidth]
       {contents/images/zisso_nagare.png}
  \caption{実装した提案手法の流れ\label{fig:nagare}}
\end{figure}

%ーーーーーーーーーーーーーーーーーーーーーーーーーーーー

\section{対象アプリとレビュー}
本研究では川面による先行研究\cite{kawatsura}のデータセットを使用するため対象アプリは先行研究のアプリに合わせている. 注意点として, BuzzVideoは2022年3月をもってサービスを終了しているため, 本研究で収集の対象となるアプリには含まれない. そのため, 本研究で新たに取得するレビューの対象となるアプリはBuzzVideo以外の12のアプリである. 
対象となっているアプリを表\ref{tb:taisyouapuri}に示す. 
\begin{table}[htbp]
  \caption{本研究の対象アプリ一覧}
  \label{tb:taisyouapuri}
  \begin{center}
  \begin{tabularx}{\linewidth}{X|l|X}
    \hline
    \mbox{アプリ名}\mbox{(一部略称)}&\mbox{Google Playストアの}\mbox{パッケージID}&\mbox{Twitterの}\mbox{検索キーワード}\\\hline\hline
    にゃんトーク&com.akvelon.meowtalk&にゃんトーク\\\hline
    スマートニュース&jp.gocro.smartnews.android&スマートニュース\\\hline
    PayPay&jp.ne.paypay.android.app&paypay\\\hline
    Coke ON&com.coke.cokeon&coke on\\\hline
    Google Fit&com.google.android.apps.fitness&google fit\\\hline
    Simeji&com.adamrocker.android.input.simeji&simeji\\\hline
    Lemon8&com.bd.nproject&lemon8\\\hline
    楽天ペイ&jp.co.rakuten.pay&楽天ペイ\\\hline
    majica&com.donki.majica&majica\\\hline
    LINE MUSIC&jp.linecorp.linemusic.android&line music\\\hline
    BuzzVideo&com.ss.android.article.topbuzzvideo&buzzvideo\\\hline
    ファミペイ&jp.co.family.familymart\verb|_|app&ファミペイ\\\hline
    CapCut&com.lemon.lvoverseas&capcut\\\hline
  \end{tabularx}\end{center}
\end{table}

%ーーーーーーーーーーーーーーーーーーーーーーーーーーーー

\section{事前準備}
本研究で対象としているアプリに関するGooglePlayストアとTwitterのレビュー及びツイートをスクレイピングし, 分析対象となるデータセットを作成する. 本研究にて使用するデータセットは先行研究によって作成された13個のアプリレビューのデータセットに加え, 本研究で新たに収集した12個のアプリレビューのデータセットを使用する. 
対象アプリを先行研究に合わせた理由としては, 現在, Twitterの利用規約によりツイートの収集可能な数に制限がある影響で, 本研究だけでは十分な数のツイートが収集できなかったためである. Twitterのツイート取得数に関する制限は\ref{sec:x}で詳しく述べる. 
そして, スクレイピングしたデータから有用な箇所を自動抽出するために一般的な自然言語処理で行われる前処理を行う. 

%ーーーーーーーーーーーーーーーーーーーーーーーーーーーー

\section{有用な箇所の自動抽出}
\subsection{概要}
本研究における自動抽出の役割は以下の2つである. 
\begin{itemize}
  \item 前処理された大量のレビューやツイートから開発に有用な文章のみを絞り込む
  \item 文章中からバグの報告やアプリに対する要望に関して記述している部分(以下 : オブジェクト)を抽出する
\end{itemize}
これにより開発に有用な情報のみが取得できる. 

\subsection{使用するモデル}
日本語のデータで事前学習済みの言語表現モデルである日本語BERTに対して質問応答形式のfune-tuningを行うことで自動抽出器を生成する. 
抽出対象となる元の文章の特徴をモデルに理解させるために質問文にその文章がGoogle PlayストアのレビューなのかTwitterのツイートなのかという「カテゴリー」の情報と「アプリ名」という2つの情報を加えることにより学習性能を上げる. 
図\ref{fig:fine-tuning}に質問応答形式によるfine-tuningのモデルを示す. 

\begin{figure}[hbtp]
  \centering
  \includegraphics[scale=0.3]
       {contents/images/fine-tuning.png}
  \caption{ファインチューニング\label{fig:fine-tuning}}
\end{figure}

また, 図\ref{fig:answer}に質問文とその答えの例を示す. 質問文にアプリの欠陥やアプリに対する要望を尋ねる文章を与え, その答えとしてオブジェクトを返すようにしている. 

\begin{figure}[hbtp]
  \centering
  \includegraphics[scale=0.4]
       {contents/images/answer.png}
  \caption{オブジェクトの抽出例\label{fig:answer}}
\end{figure}

%ーーーーーーーーーーーーーーーーーーーーーーーーーーーー

\section{クラスタリング}
\subsection{概要}
本研究におけるクラスタリングの役割は以下の2つである. 
\begin{itemize}
  \item 抽出された各オブジェクトを類似度に応じてクラスタリングする
  \item それぞれのクラスタにおける名称を決める
\end{itemize}

\subsection{クラスタリング手法}
抽出したオブジェクトのクラスタリングには先行研究のグラフクラスタリング手法\cite{sira}を参考にして実装する. 既存研究では英語の文章をベクトルに変換するためにUniversal Sentence Encoder (USE)を使用しているが, 今回は対象が日本語の文章であるため, Sentence-BERTの日本語モデルを使用する. Sentence-BERT\cite{sentence-bert}とは, 事前学習されたBERTモデルとSiamese Networkを使い, 高品質な文ベクトルを作る手法である. このモデルを使用することで, 高品質な文ベクトルが作成できる. 
したがって本研究でのクラスタリング手法は下記の手順ある. 
% \begin{enumerate}
%   \item Universal Sentence Encoder (USE)を用いて, 問題のある特徴語句を512次元のベクトルに変換
%   \item 重み付き無向グラフを構成し, 各問題機能をノードとし, 2つの問題機能のUSEベクトル間のコサイン類似度スコアをノード間の重みとする. スコアがある閾値以上の場合, 2つのノード間にエッジを追加. この閾値は入力ハイパーパラメータであり, 問題のある機能間の意味的相関を測定するために使用される. 閾値が高いほどクラスタの結束力が高まる. 
%   \item このグラフに対して, Chinese Whispers (CW)を実行し, 問題のある特徴量をクラスタリング
% \end{enumerate}
\begin{enumerate}
  \item Sentence-BERTの日本語モデルを用いて, 抽出した文章をベクトルに変換
  \item 重み付き無向グラフを構成し, 抽出したオブジェクトをノードとし, 2つの抽出した文章間のベクトル間のコサイン類似度スコアをノード間の重みとする. スコアがある閾値以上の場合, 2つのノード間にエッジを追加. この閾値は入力ハイパーパラメータであり, 抽出されたオブジェクト間の意味的相関を測定するために使用される. 閾値が高いほどクラスタの結束力が高まる. 
  \item このグラフに対して, Chinese Whispers (CW)を実行し, 問題のある特徴量をクラスタリング
\end{enumerate}

以下の図\ref{fig:clustering}にクラスタリングの概略図を示す. 丸で示されているのが抽出されたオブジェクトを表すノードであり, 点線がエッジである. このエッジは設定された閾値によって変化する. 
\begin{figure}[hbtp]
  \centering
  \includegraphics[scale=0.4]
       {contents/images/clustering.png}
  \caption{グラフクラスタリングの概略図\label{fig:clustering}}
\end{figure}

\subsection{各クラスタの名称}
クラスタを作成したら各クラスタの特徴を表す名称を決定する. 
この名称の決定にはKeyBERT\cite{keybert}を使用したキーフレーズの抽出によって実現する. KeyBERTとは, BERTの埋め込みを活用して, 文書に最も類似したキーワードとキーフレーズを作成するキーワード抽出技法である. 
具体的な手順は以下に示す通りである. 

\begin{enumerate}
  \item 文書レベルの表現を得るために, BERTを用いて文書埋め込みを抽出する
  \item N-gramの単語やフレーズについて単語埋め込みを抽出する
  \item コサイン類似度を用いて, 文書に最も類似する単語やフレーズを見つける. 最も類似している単語は, 文書全体を最もよく表現する単語として特定される
\end{enumerate}

%ーーーーーーーーーーーーーーーーーーーーーーーーーーーー

\section{画面出力・可視化}
クラスタリング結果を用いて画面出力し, 可視化を行う. クラスタごとにまとめて表示することで開発者が各レビューを確認しやすいようにする. 
開発者はアプリの修正やアップデートを行った後でユーザがどのようなレビューを挙げているかを特に確認したい. したがって, 特定の機能や期間でのレビュー内容を確認できるよう, レビューが投稿された期間やキーワードで絞り込むことが可能な検索機能を実装した. 
そして, レビューの特徴を確認するために以下2つのグラフを作成した.
\begin{itemize}
  \item 日ごとのレビュー数を表す折れ線グラフ
  \item クラスタに含まれるレビュー数の上位10個を表す棒グラフ
\end{itemize}
この2つのグラフは検索結果に応じて動的に変化するようになっている. 以上の機能をwebアプリとして実装することにより, 開発者がレビューを理解, 分析しやすいように実装されている. 
\chapter{実装}
\label{chap:zisso}

%ーーーーーーーーーーーーーーーーーーーーーーーーーーーー

\section{実装の概要}
本研究では, 膨大の数あるGooglePlayとXに投稿されたアプリレビューをスクレイピングして取得し, レビュー文に含まれるバグレポートやアプリに対する要望を示す箇所を自動抽出, その結果を利用しクラスタリング, 最後にwebブラウザ上に可視化する手法を提案する. 
\begin{description}
\item[実装環境]\mbox{}
\begin{itemize}
 \item オペレーティングシステム
    \begin{itemize}
      \item Mac OS Ventura 13.4.1
    \end{itemize}
 \item 実装言語
    \begin{itemize}
      \item Python 3.11.6
    \end{itemize}
\end{itemize}
\end{description}


\begin{figure}[hbtp]
 \centering
 \includegraphics[width=\linewidth]
      {contents/images/zisso_nagare.png}
 \caption{実装した提案手法の流れ\label{chap:nagare}}
\end{figure}



%ーーーーーーーーーーーーーーーーーーーーーーーーーーーー

\section{対象アプリとレビュー}
本研究では川面による先行研究\cite{kawatsura}のデータセットを使用するため対象アプリは先行研究のアプリに合わせている. 
今回使用するレビューのデータセットは先行研究によって作成された13個のアプリレビューのデータセットに加え, 本研究で新たに収集したアプリレビューのデータセットを使用する. 対象アプリを先行研究に合わせた理由としては, 現在, Xのポスト取得数に制限がある影響で十分な数のツイートが用意できなかったことが原因である. Xのポスト取得数の制限に関しては4.4で詳しく述べる. 
対象となっているアプリを表\ref{tb:taisyouapuri}に示す. 
\begin{table}[htbp]
  \caption{本研究の対象アプリ一覧}
  \label{tb:taisyouapuri}
  \begin{center}
  \begin{tabularx}{\linewidth}{X|l|X}
    \hline
    \mbox{アプリ名}\mbox{(一部略称)}&\mbox{Google Playストアの}\mbox{パッケージID}&\mbox{Twitterの}\mbox{検索キーワード}\\\hline\hline
    にゃんトーク&com.akvelon.meowtalk&にゃんトーク\\\hline
    スマートニュース&jp.gocro.smartnews.android&スマートニュース\\\hline
    PayPay&jp.ne.paypay.android.app&paypay\\\hline
    Coke ON&com.coke.cokeon&coke on\\\hline
    Google Fit&com.google.android.apps.fitness&google fit\\\hline
    Simeji&com.adamrocker.android.input.simeji&simeji\\\hline
    Lemon8&com.bd.nproject&lemon8\\\hline
    楽天ペイ&jp.co.rakuten.pay&楽天ペイ\\\hline
    majica&com.donki.majica&majica\\\hline
    LINE MUSIC&jp.linecorp.linemusic.android&line music\\\hline
    BuzzVideo&com.ss.android.article.topbuzzvideo&buzzvideo\\\hline
    ファミペイ&jp.co.family.familymart\verb|_|app&ファミペイ\\\hline
    CapCut&com.lemon.lvoverseas&capcut\\\hline
  \end{tabularx}\end{center}
\end{table}

%ーーーーーーーーーーーーーーーーーーーーーーーーーーーー

\section{Google Playストアのスクレイピング}
本研究で取得するレビュー情報は先行研究に合わせて下記とする. 

\begin{itemize}
 \item reviewId : レビューID
 \item userName : ユーザ名
 \item userImage : ユーザのプロフィール画像
 \item at : 投稿日時
 \item score : 星の数
 \item content : レビュー内容
 \item thumbsUpCount : このレビューが参考になったと評価した人の数
 \item reviewCreatedVersion : レビュー時のバージョン
 \item replyContent : 開発者からの返信の内容
 \item repliedAt : 開発者からの返信日時
\end{itemize}

先行研究では投稿日時が2021年10月21日〜2021年12月15日までの8週間のレビューを収集している. 用意されたGoogle Playストアの各アプリのレビュー数は表\ref{tb:rawreviewnum}の通りである. 
\begin{table}[htbp]
  \caption{収集したGoogle Playストアのレビュー数(\cite{kawatsura} p.16, 表 4.2)}
  \label{tb:rawreviewnum}
  \begin{center}
  \begin{tabular}{l|l}
    \hline
    アプリ名&収集したレビュー数(件)\\\hline\hline
    にゃんトーク&171\\\hline
    スマートニュース&1,651\\\hline
    PayPay&1,052\\\hline
    Coke ON&1,736\\\hline
    Google Fit&372\\\hline
    Simeji&468\\\hline
    Lemon8&72\\\hline
    楽天ペイ&480\\\hline
    majica&706\\\hline
    LINE MUSIC&359\\\hline
    BuzzVideo&375\\\hline
    ファミマのアプリ&290\\\hline
    CapCut&180\\\hline\hline
    合計&7,912
  \end{tabular}\end{center}
\end{table}
この先行研究のデータに加え, 本研究では2023年10月1日~12月15日のレビューを新たに取得する. 新たに取得されたGoogle Playストアの各アプリのレビュー数は表\ref{tb:rawreviewnum2023}の通りである. 
\begin{table}[htbp]
  \caption{収集したGoogle Playストアのレビュー数(2023/10/1〜12/15)}
  \label{tb:rawreviewnum2023}
  \begin{center}
  \begin{tabular}{l|l}
    \hline
    アプリ名&収集したレビュー数(件)\\\hline\hline
    にゃんトーク&\\\hline
    スマートニュース&\\\hline
    PayPay&\\\hline
    Coke ON&\\\hline
    Google Fit&\\\hline
    Simeji&\\\hline
    Lemon8&\\\hline
    楽天ペイ&\\\hline
    majica&\\\hline
    LINE MUSIC&\\\hline
    BuzzVideo&\\\hline
    ファミマのアプリ&\\\hline
    CapCut&\\\hline\hline
    合計&
  \end{tabular}\end{center}
\end{table}

Google PlayストアのレビューをスクレイピングするためにPythonのプログラムである(\verb|get_google_play_review.py|)を作成した. このプログラムの作成にあたり, Pythonのライブラリであるgoogle-play-scraperを使用する. google-play-scraperでは外部依存関係なしでPython用のGoogle Playストアを簡単にクロールするためのAPIが提供されている\cite{google-play-scraper}. 
このライブラリを使用することにより, アプリのパッケージ名, 言語, 取得する数, 順序を指定してレビューの一覧を取得することができる. 

%ーーーーーーーーーーーーーーーーーーーーーーーーーーーー

\section{Xのスクレイピング}
本研究で取得するツイート情報は先行研究に合わせて下記とする. 
\begin{itemize}
 \item id : ツイートID
 \item content : ツイート内容
 \item at : ツイート日時
\end{itemize}

まず, 先行研究で収集したツイート数を表\ref{tb:rawtweetnum}に示す. 

\begin{table}[htbp]
  \caption{収集したTwitterのツイート数(\cite{kawatsura} p.18, 表 4.3)}
  \label{tb:rawtweetnum}
  \begin{center}
  \begin{tabular}{l|l}
    \hline
    アプリ名&収集したツイート数(件)\\\hline\hline
    にゃんトーク&2,525\\\hline
    スマートニュース&50,590\\\hline
    PayPay&880,319\\\hline
    Coke ON&84,424\\\hline
    Google Fit&13,496\\\hline
    Simeji&205,327\\\hline
    Lemon8&4,376\\\hline
    楽天ペイ&11,111\\\hline
    majica&3,649\\\hline
    LINE MUSIC&184,873\\\hline
    BuzzVideo&41,656\\\hline
    ファミマのアプリ&8,867\\\hline
    CapCut&33,998\\\hline\hline
    合計&1,525,211
  \end{tabular}\end{center}
\end{table}

この先行研究に加え, 本研究では新たに2023年10月1日~12月15日のポストを取得する. Xのポスト取得に関してはTwitter APIを使用してスクレイピングを行う. Twitter APIのプランに関してはFree, Basic, Pro, Enterpriseの4つのプランが用意されておりそれぞれ料金や使用できる機能などが異なる. 大規模なサービスやビジネス向けのEnterpriseプラン以外の3つのプランの違いの一部を表\ref{tb:xplan}に示す. 
表\ref{tb:xplan}よりポストを取得するためにはBasicプラン以上に加入する必要がある. Basicプランに加入した場合でも合計で30,000件しか取得されないため本研究では先行研究のデータセットを追加で使用することとした. Xの利用規約によると, Xが提供するインターフェイスを介して行うスクレイピング以外は禁止としている. そのため, seleniumなどを使用したスクレイピングは断念し, 過去の論文のデータを使用することとした. 


\begin{table}[htbp]
  \caption{プランとできること}
  \label{tb:xplan}
  \begin{center}
  \begin{tabular}{|l|c|c|c|}
    \hline
    &Free&Basic&Pro \\\hline\hline
    料金&無料&月額100ドル&月額5,000ドル \\\hline
    月間ポスト数の上限&1,500&3,000&300,000 \\\hline
    月間ポスト取得数&0&10,000&1,000,000 \\\hline
  \end{tabular}\end{center}
\end{table}

Twitter APIを使用してポストを取得するためにPythonのプログラムである(\verb|get_tweet.py|)を作成した. このプログラムではTwitter APIにアクセスするためのライブラリであるTweepy\cite{tweepy}を使用した. まずAPIキーなどの4つの認証情報をセットする. 次にClientクラスのsearch\_resent\_tweetメソッドを使用してツイートを取得する. 
このメソッドは最大過去7日間まで遡ってツイートを取得できる. search\_all\_tweetsメソッドでは全てのツイートを取得できるが, ``Academic Research''という学術用の用途でAPI承認されたユーザーしか使用できないため本研究では使用しなかった. 
先行研究と同じ情報を取得するために, 本研究では引数として以下のものを与えた. 
\begin{itemize}
 \item max\_resul : 検索結果の最大数. 10〜100の数値で, デフォルトは10
 \item query: 検索ワード
 \item tweet\_field: ツイートフィールドを選択. 今回はツイート日時を取得するために["created\_at"]とした. 
 \item end\_time: 期間の終わりを指定できる(UTCタイムスタンプ)
\end{itemize}

新たに取得したXのポスト数を表\ref{tb:rawtweetnum2023}に示す

\begin{table}[htbp]
  \caption{収集したXのポスト数}
  \label{tb:rawtweetnum2023}
  \begin{center}
  \begin{tabular}{l|l}
    \hline
    アプリ名&収集したツイート数(件)\\\hline\hline
    にゃんトーク&\\\hline
    スマートニュース&\\\hline
    PayPay&\\\hline
    Coke ON&\\\hline
    Google Fit&\\\hline
    Simeji&\\\hline
    Lemon8&\\\hline
    楽天ペイ&\\\hline
    majica&\\\hline
    LINE MUSIC&\\\hline
    BuzzVideo&\\\hline
    ファミマのアプリ&\\\hline
    CapCut&\\\hline\hline
    合計&
  \end{tabular}\end{center}
\end{table}

%ーーーーーーーーーーーーーーーーーーーーーーーーーーーー

\section{前処理}
機械学習によるレビューに含まれる有用な箇所の自動抽出の精度を上げるために, GooglePlayストアとXから取得したデータに対して前処理を行うプログラム(\verb|preprocessing_google.py|, \verb|preprocessing_twitter.py|)を作成した. この処理では以下に示す処理を行う. この処理は一般的な自然言語処理の手法を参考としている. 
\begin{itemize}
  \item 英語を全て小文字に揃える. 
  \item 以下の文字列を削除. 
    \begin{itemize}
      \item 「」【】()()『』
      \item @@から始まるメンション
      \item \#から始まるタグ
      \item URL
      \item 半角空白,全角空白
      \item 絵文字
      \item 日本語を含まないレビュー
    \end{itemize}
  \item レビューやツイートには, 異なるバグの報告や新しい機能の要望に関する文が2文以上からなるものがある. そのため, 「。」「.」「!」「!」「?」「!」「\verb|\n|」「\verb|\r\n|」でそれぞれの文に分割する. 
\end{itemize}
以下の図\ref{chap:preprocessing}がレビューを前処理した例である. 2つの文で構成されているため「。」で区切り分割している. また絵文字は削除されている. 

\begin{figure}[hbtp]
 \centering
 \includegraphics[scale=0.5]
      {contents/images/preprocessing.png}
 \caption{前処理の例\label{chap:preprocessing}}
\end{figure}

前処理した結果をcsvファイルにて保存する. 保存する項目としては, 投稿日時(at), レビューのid(reviewId)またはツイートid(id), そして, 前処理した文章である. 図\ref{tb:googlecsv}, 図\ref{tb:twittercsv}に前処理結果後のcsvファイルの一部を示す. 

\begin{table}[htbp]
  \caption{Google Playストアレビューの前処理結果(buzzvideo)}
  \label{tb:googlecsv}
  \begin{center}
  \begin{tabularx}{\linewidth}{|l|l|X|}
    \hline
    at&reviewId&content\\\hline\hline
    2021-12-15 19:25:30&gp:AOqpTOHj6w ...&バズビデオを見て、感動をありがとう\\\hline
    2021-12-15 12:28:09&gp:AOqpTOHleV ...&内容が残酷で異常な人が多い\\\hline
    2021-12-15 11:09:50&gp:AOqpTOHG7O ...&分かりづらい\\\hline
    2021-12-14 15:16:33&gp:AOqpTOGWvT ...&ばず29さいって人が投稿してる動画すべて虚偽動画なのでアカウント削除と動画削除して欲しい\\\hline
    2021-12-14 15:16:33&gp:AOqpTOGWvT ...&あるだけで大迷惑です\\\hline
    2021-12-14 15:16:33&gp:AOqpTOGWvT ...&二度と登録し直せないよう個体識別番号で縛ってください\\\hline
    2021-12-14 15:16:33&gp:AOqpTOGWvT ...&お願いします\\\hline
  \end{tabularx}\end{center}
\end{table}

\begin{table}[htbp]
  \caption{ツイートの前処理結果(BuzzVideo)}
  \label{tb:twittercsv}
  \begin{center}
  \begin{tabularx}{\linewidth}{|l|l|X|}
    \hline
    at&id&content\\\hline\hline
    2021-12-15T23:55:11.000Z&1471267626655825922&芸能人に似てる気がするけど名前が思い出せない\\\hline
    2021-12-15T23:53:43.000Z&1471267256659509249&驚愕男性が豆乳を飲むべき3つの理由\\\hline
    2021-12-15T23:53:43.000Z&1471267256659509249&男だからこそ注目したい豆乳のメリットとは\\\hline
    2021-12-15T23:53:17.000Z&1471267149746679813&感情を乗せた歌声と歌詞に聞き惚れちゃう♪壊れかけのradio\\\hline
    2021-12-15T23:53:10.000Z&1471267120264904705&kkと眞子の酷い嘘\\\hline
    2021-12-15T23:53:10.000Z&1471267120264904705&恐ろしい真実が明らかに\\\hline
  \end{tabularx}\end{center}
\end{table}

%ーーーーーーーーーーーーーーーーーーーーーーーーーーーー

\section{有用な箇所の自動抽出}
\subsection{概要}
膨大な数あるレビュー文から開発に有用なレビューを絞り込み, かつ有用なレビュー文の中からバグの報告やアプリに対する要望に関して記述している部分を自動抽出する. 自動抽出のために日本語のデータで事前学習済みの言語表現モデルである日本語BERTに対して質問応答形式のfune-tuningを行うことで自動抽出器を生成する. 
抽出を行う文章の特徴をモデルに理解させるために質問文にその文章がGooglePlayストアのレビューなのかXの文章なのかという「カテゴリー」の情報と「アプリ名」という2つの情報を加える. 図\ref{chap:fine-tuning}に質問応答形式によるfine-tuningのモデルを示す. 
\begin{figure}[hbtp]
  \centering
  \includegraphics[scale=0.3]
       {contents/images/fine-tuning.png}
  \caption{前処理の例\label{chap:fine-tuning}}
 \end{figure}

また, 図\ref{chap:answer}に質問文とその答えの例を示す. 質問文にアプリの欠陥やアプリに対する要望を尋ねる文章を与え, その答えとして欠陥や要望を示す箇所を返すようにしている. 
\begin{figure}[hbtp]
  \centering
  \includegraphics[scale=0.4]
       {contents/images/answer.png}
  \caption{前処理の例\label{chap:answer}}
 \end{figure}
\subsection{データセット}
データセットとして, Google PlayストアとXのツイートからそれぞれ5,000件ずつ合計で10,000件のデータをランダムに抽出し手作業で有用な箇所を抽出した. データセットの作成は情報工学科の学部4年生2人がそれぞれ手作業で行い, お互いの抽出した箇所が異なっていたものは議論することにより決定した. 
データセットには前処理したデータを利用したcsvファイルを使用する. このcsvファイルにはid,アプリ名,投稿日時,本文, 手動で抽出した結果の5つの情報が入っている. idは前処理したデータを識別するために与えられ, Google Playストアのレビュー文はg\_{index}, twitterのツイートのidはt\_{index}とする. 
10,000件のデータセットのうち, 6,000件を訓練データ, 2,000件を検証用データ, 2,000件をテストデータとする. 質問応答形式のfine-tuningを行うために, csv形式であるデータセットをソースコード\ref{json}に示すようにjson形式に変換する. 

\begin{lstlisting}[caption=データセット.json,label=json]
  {
    "version": "v2.0", 
    "data": [
      {
        "title": "モバイルアプリのレビュー", 
        "paragraphs": [
          {
            "context": "本アカウントのフォローやリツイートお願いします",
            "qas": [
              {
                "id": "t_2223388",
                "question": "この文はTwitterのツイートです。
                             paypayアプリの欠陥やpaypayアプリに対する
                             要望が書かれているのはどこですか?",
                "is_impossible": true,
                "plausible_answers": [{"text": "", "answer_start": -1}],
                "answers": [{"text": "", "answer_start": -1}]
              }
            ]
          },
          {
            "context": "11/25前後からアプリを開いても強制終了、
                      会員バーコードもクーポンも何も出せない状態、
                      これでは買い物ができないと、こちらのレビュー
                      を見に来て沢山の方が同じ状態であることが
                      わかった",
            "qas": [
              {
                "id": "g_6041", 
                "question": "この文章はGooglePlayストアのレビューです。
                            majicaアプリの欠陥やmajicaアプリに対する
                            要望が書かれているのはどこですか?",
                "answers": [{"text": "アプリを開いても強制終了、
                                      会員バーコードもクーポン
                                      も何も出せない", 
                             "answer_start": -1}], 
                "is_impossible": false
              }
            ]
          }, ...
        ]
      }
    ]
  } 
\end{lstlisting}

このjsonファイルは下記の構成になっている. 
\begin{itemize}
  \item version: バージョンを表す. 今回は答えられない質問を含むSQuAD 2.0と同じバージョンのため, v2.0とする
  \item title: contextのタイトル
  \item paragraphs: context1つとそれに関連する質問, 答えがリスト形式で保持されている
  \item qas: 質問と回答がリスト形式となっている
  \item context: 元の文章(抽出する前の文章)
  \item id: 設定したid
  \item question: 質問文
  \item is\_impossible: 答えられない質問ならtrue, それ以外はfalse
  \item plausible\_answers: 質問が答えられない時のみ存在し, 問題文から答えになりうる部分を抽出
  \item answers: contextから抜き出した答えとその位置情報がリスト形式で保持されている. 答えを複数用意することもできる. 
  \item text: contextから抜き出した答えのテキスト情報(抽出する文章)
  \item answe\_start: contextから抜き出した答えの位置情報
\end{itemize}

\subsection{モデルのfine-tuning}
用意したデータセットを用いてモデルをfine-tuningする. 本研究では事前学習済みモデルを提供するフレームワークであるHuging FaceのTransformersを通して利用できる東北大学のモデル\cite{tohoku}を使用する. このモデルは日本語のWikipediaのデータを用いて学習されている\cite{tohoku}. 
この東北大学が公開している日本語BERTのうち, whole word maskingを適用して学習させているモデル\cite{masking}を用いる. whole word maskingとは事前学習時に単語ごとでマスクするかどうかを決め, マスクする単語に対応するサブワードを全てマスクする方式である. モデルのパラメータは以下に示す通りである. 
\begin{itemize}
  \item 学習率: 3e-5
  \item エポック数: 10
  \item バッチサイズ: 12
\end{itemize}

実装にはTransformersに含まれるスクリプトであるrun\_squad.pyを用いる. 

\subsection{自動抽出}
fine-tuningを行ったモデルを使用して自動抽出を行う. 前処理したGooglePlayストアのデータ14,051件と, twitterのデータ4,634,319件のデータから有用な箇所を自動抽出する. これにより開発に有用なレビューのみを選択でき, その文章の中に含まれるバグの報告やアプリに対する要望に関して記述している部分を抽出できる. 

結果は表\ref{tb:googleqa}に示すようにcsv形式で保存する. 

\begin{table}[htbp]
  \caption{Google Playストアレビューの自動抽出結果(google\_fit)}
  \label{tb:googleqa}
  \small
  \begin{center}
  \begin{tabularx}{\linewidth}{|l|l|X|X|X|}
    \hline
    id&app\_name&datetime&context&prediction\\\hline\hline
    g\_955&coke\_on&2021-11-27 11:17:03&商品が出ない事が何回か発生しました&商品が出ない\\\hline
    g\_956&coke\_on&2021-11-02 12:15:37&使用している端末が、利用できる端末の一覧表にないため、サポートは期待できない&使用している端末が、利用できる端末の一覧表にない\\\hline
    g\_959&coke\_on&2021-11-11 15:32:50&そもそも自販機側が黄色点滅していなくて買えないことが多過ぎです&自販機側が黄色点滅していなくて買えない\\\hline
    g\_961&coke\_on&2021-11-14 23:13:26&今まではcoke\_on対応を優先してかっていたが、これからはコカコーラ製品全般をできるだけ買わないようにする&今まではcoke\_on対応を優先してかっていたが、これからはコカコーラ製品全般をできるだけ買わないようにする\\\hline
    g\_964&coke\_on&2021-10-24 12:30:07&自販機との接続を早くしてほしい&自販機との接続を早くしてほしい\\\hline
    g\_965&coke\_on&2021-11-11 16:43:39&やっと繋がっても先にキャンペーン広告が出てすぐに買えないのが不親切&キャンペーン広告が出てすぐに買えない\\\hline
    g\_969&coke\_on&2021-12-07 08:28:27&コークオンパスのフリー20プランの残り回数が分かりやすく表示してほしい&コークオンパスのフリー20プランの残り回数が分かりやすく表示してほしい\\\hline
    g\_973&coke\_on&2021-11-23 14:51:45&反応しない&反応しない\\\hline
    g\_977&coke\_on&2021-11-11 18:22:04&2本以上の購入はとても使えません&2本以上の購入はとても使えません\\\hline
    g\_978&coke\_on&2021-11-01 11:06:09&それと対応自販機との連携が悪い&自販機との連携が悪い\\\hline
    g\_980&coke\_on&2021-10-22 03:22:06&動きが遅い&動きが遅い\\\hline
    g\_982&coke\_on&2021-11-22 11:58:45&ただ自分のスマホのストレージが小さく、データ容量が大きいため今回一先ず削除いたします&自分のスマホのストレージが小さく、データ容量が大きい\\\hline
  \end{tabularx}\end{center}
\end{table}
%ーーーーーーーーーーーーーーーーーーーーーーーーーーーー

\section{クラスタリング}
抽出した文章のクラスタリングには既存研究のクラスタリング手法\cite{sira}を参考にして実行する. この手法は以下の3つの手順である. 
\begin{enumerate}
  \item Universal Sentence Encoder (USE)を用いて, 問題のある特徴語句を512次元のベクトルに変換
  \item 重み付き無向グラフを構成し, 各問題機能をノードとし, 2つの問題機能のUSEベクトル間のコサイン類似度スコア(比率)をノード間の重みとする. 比率は入力ハイパーパラメータであり, 問題のある機能間の意味的相関を測定. ハイパーチューニングした閾値は0.5とした
  \item このグラフに対して, Chinese Whispers (CW)を実行し, 問題のある特徴量をクラスタリング
\end{enumerate}

既存研究では英語の文章をベクトルに変換するためにUniversal Sentence Encoder (USE)を使用しているが, 今回は対象が日本語の文章であるため, Sentence-BERTの日本語モデルを使用する. Sentence-BERTとは, 事前学習されたBERTモデルとSiamese Networkを使い, 高品質な文ベクトルを作る手法である. このモデルを使用することで, 高品質な文ベクトルが作成できる. 
したがって本研究でのクラスタリング手法は下記である. 
\begin{enumerate}
  \item Sentence-BERTの日本語モデルを用いて, 抽出した文章をベクトルに変換
  \item 重み付き無向グラフを構成し, 各問題機能をノードとし, 2つの抽出した文章間のベクトル間のコサイン類似度スコア(比率)をノード間の重みとする. 比率は入力ハイパーパラメータであり, 抽出した文章間の意味的相関を測定. 検証した結果, 閾値は0.8とした
  \item このグラフに対して, Chinese Whispers (CW)を実行し, 問題のある特徴量をクラスタリング
\end{enumerate}

これにより, 抽出した文章をその文章が示す意味に応じてクラスタリングすることができる. それぞれの文章にクラスタの番号(以下 : クラスタ番号)が振られ, クラスタ番号が同じものが同じクラスタとなり番号が近いものは意味的相関が近いことを表す. 

結果はcsvファイルに保存される. 表\ref{tb:clustering}に示されるように抽出した文章にクラスタ番号が振られる. 

\begin{table}[htbp]
  \caption{抽出した文章とクラスタ番号(google\_fit)}
  \label{tb:clustering}
  \begin{center}
  \begin{tabularx}{\linewidth}{|X|c|}
    \hline
    prediction&cluster\\\hline\hline
    十分歩いて108歩とかふざけんな&273\\\hline
    278歩に減っていた&274\\\hline
    再起動しても直らない&275\\\hline
    接続/連携を適宜確認しておく必要がある&276\\\hline
    歩いた歩数より足りない&279\\\hline
    使えない&280\\\hline
    使えない&280\\\hline
    歩けない&280\\\hline
    使えない&280\\\hline
    動かなかった&280\\\hline
    反応しない&280\\\hline
    使い方も分からない&280\\\hline
    使えない&280\\\hline
    使えない&280\\\hline
    動かなくなった&280\\\hline
    長期放置されてるんでしょうか&281\\\hline
    下がるって何故でしょうか&282\\\hline
    カウントされなくなる&283\\\hline
    何もカウントしなくなった&283\\\hline
    カウント出来ていない&283\\\hline
    カウントされず&283\\\hline
    記録ができませんと&283\\\hline
    計測しなくなった&283\\\hline
    カウントされなくなった&283\\\hline
    カウントしない&283\\\hline
    全くカウントされていない&283\\\hline
    カウントしなくなりました&283\\\hline
    データが反映されなくなった&283\\\hline
  \end{tabularx}\end{center}
\end{table}


%ーーーーーーーーーーーーーーーーーーーーーーーーーーーー

\section{画面出力・可視化}
クラスタリングした結果をwebブラウザ上で可視化する. webアプリケーションの実装に使用した言語, フレームワークは以下となっている. 
\begin{itemize}
    \item フロントエンド: HTML/CSS, JavaScript
    \item バックエンド: Python
    \item フレームワーク: Flask
\end{itemize}

アコーディオンメニューを使用して抽出した文章の一覧をクラスタごとに表示する. また, 抽出した文章をクリックすると投稿日時や元のレビュー文がモーダルウィンドウで表示する. 
さらにレビューが投稿された期間やレビュー文のキーワードで絞り込み検索することが可能となっている. そして, 日ごとのレビュー数を表す折れ線グラフとクラスタに含まれるレビュー数の上位10個を表す棒グラフを作成した. この2つのグラフは検索結果に応じて動的に変化するようになっている. 
\chapter{評価}
\label{chap:kekkahyouka}

\section{RQ1:レビューの抽出性能はどの程度か}
\subsection{結果}
自動抽出器の抽出性能について調査する. テストデータ2,000件に対して自動抽出を行い手動で抽出した結果と比較した. この自動抽出に関しては, レビュー文中に有用な情報がないものは答えのない文章と判断し, レビュー文中に有用な情報がある場合は答えのある文章と判断し, レビュー文の中にある有用な情報を答えとする. 
答えがある問題に関しては抽出した結果が完全に一致している場合, 部分的に一致している場合, 全く一致していない場合の3種類を考える. 
結果を以下の表\ref{tb:qa}に示す. 

\begin{table}[htbp]
  \caption{抽出結果}
  \label{tb:qa}
  \begin{center}
  \begin{tabularx}{\linewidth}{|X|X|}
    \hline
    答えがある問題数&468\\\hline
    答えがある問題の正当数&261\\\hline
    答えがある問題の部分一致正答数&124\\\hline
    答えがない問題数&1532\\\hline
    答えがない問題の正答数&1483\\\hline
    \hline
    答えがある問題の正答率&55.8\%\\\hline
    答えがある問題の部分一致を含めた正答数&82.3\%\\\hline
    答えがない問題の正答率&96.8\%\\\hline
    \hline
    全体の正答率&87.2\%\\\hline
    部分一致を含めた全体の正答率&93.4\%\\\hline
  \end{tabularx}\end{center}
\end{table}

答えがない問題の正答率は96.8\%と非常に精度の高い結果となった. すなわち, そのレビュー文に有用な情報があるかどうかを判別する精度が高いことが示された. 
一方で, 答えがある問題の正答率は55.8\%とあまり精度が高くないものの部分一致を含めた正答率は, 82.1\%と比較的高い結果が得られた. 

\subsection{考察}
正答率が下がってしまった原因に関して考察する. 
答えがある問題, すなわち有用な箇所が存在する文章のうち誤って抽出してしまった例を以下の表\ref{tb:mistake}に示す.

\begin{table}[htbp]
  \caption{誤答となったテスト結果(答えあり)}
  \label{tb:mistake}
  \begin{center}
  \begin{tabularx}{\linewidth}{|X|X|X|}
    \hline
    元の文章&自動抽出した回答&手動抽出した回答\\\hline\hline
    店で使えずに仕方なく現金で払いました&&店で使えず\\\hline
    とにかく地図検索がクソ&とにかく地図検索がクソ&地図検索がクソ\\\hline
    急に曲が止まって何しても流れないから端末再起動させたらようやく流れた&急に曲が止まって何しても流れないから端末再起動させたらようやく流れた&急に曲が止まって何しても流れない\\\hline
    データのカウント数が減ることがある&データのカウント数が減る&カウント数が減ることがある\\\hline
    ウォーキングで設定してるのに10分の1しかカウントしない&ウォーキングで設定してるのに10分の1しかカウントしない&10分の1しかカウントしない\\\hline
    類似アプリと比べて突出した利点もなく、あえてlinemusicを選ぶ理由がない&&類似アプリと比べて突出した利点もなく\\\hline
    コークオンを使おうとしてるのに、使えないなんでか、解らない&コークオンを使おうとしてるのに、使えないなんでか&コークオンを使おうとしてるのに、使えない\\\hline
    12月1日正午再び開けなくなりバーコードのみ表示&再び開けなくなりバーコードのみ表示&開けなくなりバーコードのみ表示\\\hline
    クレジットカードがjcb縛りなんて驚愕です&&クレジットカードがjcb縛り\\\hline
    機内モードにすることで会員バーコードは表示できるのでチャージや支払いは出来るが、クーポンを取得したりできないので不便&クーポンを取得したりできない&機内モードにすることで会員バーコードは表示できるのでチャージや支払いは出来るが、クーポンを取得したりできないので不便\\\hline
  \end{tabularx}\end{center}
\end{table}

次に答えのない問題, すなわち有用な箇所が存在しないレビューにも関わらず誤って抽出してしまった例を以下の表\ref{tb:mistake2}に示す.
\begin{table}[htbp]
  \caption{誤答となったテスト結果(答えなし)}
  \label{tb:mistake2}
  \begin{center}
  \begin{tabularx}{\linewidth}{|X|X|X|}
    \hline
    元の文章&自動抽出した回答&手動抽出した回答\\\hline\hline
    初めてはどきどきする&初めてはどきどきする&\\\hline
    wifiも4gも正常です&wifiも4gも正常です&\\\hline
    ふざけたアプリ&ふざけたアプリ&\\\hline
    改善策はあるのでしょうか&改善策はあるのでしょうか&\\\hline
    自分でインターネットで調べるまで、今回の不具合についての更新がある事を知りませんでした&不具合&\\\hline
    原因が分かりません、機種変してからか&原因が分かりません&\\\hline
    短いタイトルで読んでみたいと思うので、今後もわかりやすい記事を期待します&今後もわかりやすい記事を期待します&\\\hline
  \end{tabularx}\end{center}
\end{table}

このような結果から正答率の減少となる原因はいくつか考えられる. 

まず, 手動で抽出した回答の精度に問題があることがわかる. データセットは情報工学科の学部4年生の2名で作成したものだが, このデータセットの精度に問題があることが正答率の減少につながっていると考えられる. 

例えば, 表\ref{tb:mistake}にある「類似アプリと比べて突出した利点もなく、あえてlinemusicを選ぶ理由がない」というレビュー文に対して, 手動で「類似アプリと比べて突出した利点もなく」を抽出しているけれど, これはアプリの欠陥でもアプリに対する要望でもないため本来抽出するべきでない. 

また, 表\ref{tb:mistake2}にある「短いタイトルで読んでみたいと思うので、今後もわかりやすい記事を期待します」というレビュー文に対して, 「今後もわかりやすい記事を期待します」はアプリに対する要望にも関わらず, 手動では抽出していない. このように手動での抽出精度を上げることによって精度の向上が見られると考えられる. 

次にレビュー文の特徴が精度に影響していると考えられる. レビュー文は短く構造化されていないため, 内容が曖昧であったり, 意図が明確でないレビューが見受けられる. したがって, このような文章から答えを抽出するのは難易度が高い. したがってレビュー文が精度に影響を与えている. 

そして, 自動抽出した結果と手動で抽出した結果を比較すると, 自動抽出した結果の方が答え(有用な箇所)を長めに抽出する傾向にあることがわかる. 
例えば, 「12月1日正午再び開けなくなりバーコードのみ表示」というレビュー文に対して, 自動抽出では「再び開けなくなりバーコードのみ表示」となり, 手動では「開けなくなりバーコードのみ表示」となっている. 

このように, 自動抽出した結果に関しては副詞や形容詞のトークンを回答に含める傾向があるため抽出する箇所が長くなる. しかし, このような単語を含めるかどうかはこの後のクラスタリングに大きな影響を与えないためそういった単語を排除するようモデルに学習させる必要はないと考えられる. 

今回のタスクは質問応答形式のタスクの中でも, 答えが名詞や動詞などの1つのトークンだけではなく, 複数のトークンを含めることが多いため抽出する人や抽出器によって結果が変わる難易度の高いタスクである. 


\subsection{総括}
生成した自動抽出器は, レビュー文中に開発に有用な情報があるかどうかを判別する精度がかなり高いことが示された. 
また, レビュー中から有用な情報を抽出する精度に関しては, 手動の抽出結果と完全に一致させることは難しいことが結果からわかる. ただ, 2つの抽出結果を比較してみると, 副詞や形容詞などの単語が含まれているかどうかだけの違いなど, ほぼ手動で抽出した結果と変わりないものも多く存在する. そのため部分一致の精度は高いことが結果からわかる. 

そして, 手動で抽出した箇所が誤っている場合も多く, これが自動抽出器の精度や正答率に影響を与えることがわかった. 実際に自動抽出した結果の方が正しいものもいくつか存在した. 課題としては手動で作成したデータセットの精度を上げることである. 

%ーーーーーーーーーーーーーーーーーーーーーーーーーーーー

\section{RQ2:クラスタリングの性能はどの程度か}
\subsection{評価手法}
クラスタリングの性能について調査する. クラスタリングの性能評価にはARI(Adjusted Rand Index)を用いる. -1〜1の値を取り, 2つのクラスターの一致度合いを計測する. 
ARIの計算にはRI(Rand Index)の値を用いる. RIは式(\ref{eq:ri})に示されるように計算される. 

\begin{equation}
  \label{eq:ri}
  RI = \frac{a+b}{\binom{n}{2}}
\end{equation}

ここで, \(a\)は予測されたクラスタリング結果とGround-truthのクラスタリング結果で同じクラスタに割り振られるペアの数を表し, \(b\)は異なるクラスタに割り振られるペアの数を表す. \(\binom{n}{2}\)はn個の抽出したオブジェクトの集合において順序のないペアの総数である. 
RIでは, 2つのクラスタリングに相関がない場合でも, 高い値を取ってしまう. したがってARIでは相関のないクラスタリングにペナルティを与える. このペナルティは「相関のない(独立な)クラスタリングをした時のRIの値」である. ARIは式(\ref{eq:ari})に示されるように計算される. 

\begin{equation}
  \label{eq:ari}
  ARI = \frac{RI-E(RI)}{max(RI)-E(RI)}
\end{equation}

\(E(RI)\)はRIの期待値となっている. このように計算することによって, クラスタ数やサンプル数に関係なく, ランダムなクラスタリングでは0に近い値を持つことが保証されている. 
クラスタリング性能を評価するために, GooglePlayストアのレビューから自動抽出したオブジェクト166件を手動でクラスタリングし, 正解データを作成した. 本研究ではこの正解データとのARIを算出し精度を確認することとする. 

\subsection{閾値ごとの結果}
本研究で実装しているグラフクラスタリングでは, 設定する閾値に応じて作成されるグラフが変わるため, クラスタリングの結果が大きく変わる. したがって, 閾値ごとのARIの結果を比較し最もARIが高くなる閾値を見つける必要がある. 
閾値とARIの関係を以下のグラフ\ref{fig:cw_graph}に示す.

\begin{figure}[hbtp]
  \centering
  \includegraphics[scale=0.8]
    {contents/images/cw_graph.png}
  \caption{閾値ごとのARIの結果\label{fig:cw_graph}}
\end{figure}

このグラフより, 閾値を0.8としたときのARIが最も高くなることがわかった. したがって本研究では全てのデータのクラスタリングにおいて閾値は0.8に設定して行った. 

\subsection{他の手法との比較}
本研究で実装したChinese Whispersと, 一般的に使用されているクラスタリングであるK-Means, 階層型クラスタリングを比較することにより本研究のクラスタリング手法の性能の高さを示す. 

K-Meansは非階層型クラスタリングのアルゴリズムである. まず, 互いのデータをランダムなクラスに配置したのちにクラスごとの重心を計算する. そして, 各データに対して重心が最も近いクラスタを割り振り重心を再計算する. このステップを重心が動かなくなるまで繰り返すことによりクラスを決定する.

階層型クラスタリングはデータからクラスタの階層構造を抽出する手法である. 最初は各データがそれぞれ1つのクラスタを持つ. そしてクラスタ間の距離が最も近い2つのクラスタを1つのクラスタにまとめる. これを繰り返していき, クラスタを大きくしていく. クラスタ間距離の計算方法はウォード法や群平均法, 最短距離法, 最長距離法などいくつかの方法がある. 

K-Means, 階層型クラスタリングは共にクラスタ数を事前に選択する必要があるためクラスタ数をいくつに設定すると最もARIが高くなるか検証した. 
クラスタ数とARIの関係を以下の図\ref{fig:kmeans_graph}, \ref{fig:agg_graph}に示す.

\begin{figure}[hbtp]
  \centering
  \includegraphics[scale=0.8]
    {contents/images/kmeans_graph.png}
  \caption{K-Meansにおけるクラスタ数ごとのARIの結果\label{fig:kmeans_graph}}
\end{figure}

\begin{figure}[hbtp]
  \centering
  \includegraphics[scale=0.8]
    {contents/images/agg_graph.png}
  \caption{階層型クラスタリングにおけるクラスタ数ごとのARIの結果\label{fig:agg_graph}}
\end{figure}

検証した結果, K-Meansではクラスタ数が119, 階層型クラスタリングではクラスタ数が129の時にそれぞれARIは最も高い値を示した. 次にK-Means, 階層型クラスタリングとChinese Whispersの結果を比較した結果が以下の表\ref{tb:two_ari}である. 

\begin{table}[htbp]
  \caption{2つの手法におけるARI}
  \label{tb:two_ari}
  \begin{center}
  \begin{tabularx}{\linewidth}{|X|X|}
    \hline
    手法&ARIの最大値\\\hline\hline
    階層型クラスタリング&0.33581598364914955\\\hline
    K-Means&0.36232268008959323\\\hline
    Chinese Whispers&\textbf{0.5480604134843943}\\\hline
  \end{tabularx}\end{center}
\end{table}

比較した結果, Chinese WhispersのARIの最大値はK-Meansよりも約0.18, 階層型クラスタリングよりも約0.21ほど高いことが示された. 

\subsection{総評}
Chinese Whispersで作成されるグラフの閾値について検証した結果, 閾値を0.8としたときに最もARIが高い値を示すことがわかった. 
また, K-MeansとARIの最大値を比較した結果, Chinese Whispersの方がARIが高いことから本研究のクラスタリングの精度の高さを示すことができた. 
%ーーーーーーーーーーーーーーーーーーーーーーーーーーーー

% \section{RQ3:付与した各クラスタの名称は適切か}
% クラスタリング後の各クラスタの名称が適切なものとなっているか調査する. 

%ーーーーーーーーーーーーーーーーーーーーーーーーーーーー

\section{RQ3:可視化ツールの有用性はどうか}
\subsection{概要}
抽出, クラスタリングによって得られた結果を表示する可視化ツールの有用性について示す. 一般的にレビューの閲覧や確認で使用されるGooglePlayストアのレビュー欄と比較して, 本研究で作成したwebサイトがレビューの分析にどのように役立つのかを記述する. 

\subsection{GooglePlayストアのレビュー欄}
GooglePlayストアのレビュー欄は以下の図\ref{fig:google_play}, 図\ref{fig:google_play_graph}のような構成となっている. 
\begin{figure}[hbtp]
  \centering
  \includegraphics[scale=0.3]
    {contents/images/google_play.png}
  \caption{GooglePlayストアのレビュー欄\label{fig:google_play}}
\end{figure}

\begin{figure}[hbtp]
  \centering
  \includegraphics[scale=0.3]
    {contents/images/google_play_graph.png}
  \caption{各評価の数を表すグラフ\label{fig:google_play_graph}}
\end{figure}

GooglePlayストアのレビュー欄では評価(星1〜星5)がそれぞれいくつ付けられているのかを表示するグラフ(図\ref{fig:google_play_graph})と, 各レビューが記載されている. 各レビューには何人のユーザが役に立ったかが記載されており, これによりユーザーから見た各レビューの評価がわかる. 
また, 主な機能として評価ごとの絞り込み, 評価や時系列による並び替えがある. 

以下ではGooglePlayストアのレビュー欄と本研究のwebアプリの違いを示すことで本研究のwebアプリの有用性を示す. 

\subsection{キーワードと期間の検索}
このwebアプリではキーワードの検索により特定の機能に関するレビューを絞り込むことができる. また, 期間を絞り込むことができるため特定の期間に投稿されたレビューだけを表示することができる.  

この機能は特にアプリのアップデート時に活用できる. アプリのアップデート時に開発者は改善した機能や新しく追加した機能が正常に動いているかどうか, アップデートの前後でレビュー数に変化があるかどうかを知りたい. そのためこの検索機能を用いて分析対象となるアプリを絞り込み表示することで確認するレビュー数を大幅に減らすことができる. 

\subsection{アプリ間のレビュー比較}
このwebサイトでは一覧画面にてアプリごとのレビュー数の推移を表示することができる. そのためアプリ間でレビュー数にどのような変化があるのかを確認することができる. 

この機能を用いて, 類似した機能を持つアプリがどのような問題を抱えているかをレビューを確認することによって分析することができる. 類似したアプリで見つけられた欠陥は自身の開発しているアプリで同じような欠陥が見つかる可能性がある. 
したがって開発者はこのwebサイトで他のアプリのレビューを確認することにより自身の開発に活用することができる. 

\subsection{レビューの分析時間短縮}
GooglePlayストアのレビュー欄の欠点は開発に役立たないレビューが多く存在し, 似たような機能の記述がまとまっていないため開発者が分析するのには時間がかかることである. 
このような欠点を解消したのが本研究のwebサイトである. まず, 開発に役立つレビューかどうかをフィルタリングして類似したレビューをまとめて表示している. そのため開発者はレビューを分析する時間を短縮できる. 
また, GooglePlayストアのレビュー欄ではどの期間に多くのレビューが投稿されたかを確認するのに時間がかかる. しかし, 本研究のwebサイトであれば日ごとのレビュー数の推移をグラフにより表示しているためどの期間に多くのレビューが書かれているかを確認できる. 
このように本研究のwebサイトを使用することにより, レビューの分析や閲覧にかかる時間を大幅に短縮することが可能となる. 

\subsection{総評}
以上で示したように本研究で作成されたwebサイトを使用することにより開発者がレビューを確認する時間の短縮や効率の向上につながっている. レビュー分析を効率化させることは開発者にとって1つの課題であるため本研究の可視化ツールは非常に有用であると考えられる. 

\section{妥当性の脅威}
\subsection{対象アプリ}
本研究の対象アプリはデータ取得の問題により先行研究のアプリに合わせているが, カテゴリーなどが統一されていない. そのため, 提案手法の一般性は確保されていない. 
したがって, 様々なアプリのレビューデータに提案手法を実行して, 有効性を確認する必要がある. 

\subsection{トレーニングデータ}
本研究では情報工学科の学生2人により10,000件のレビューからトレーニングデータを作成した. RQ1でも述べたように, このトレーニングデータの精度には改善の余地がある. 
トレーニングデータの精度を向上させるために, アプリの開発者などレビューに関する専門的知識を持つ人がトレーニングデータを作成する, トレーニングデータを作成する人数を増やして議論するなど工夫が必要である. 
また, 本研究ではトレーニングデータの数を増やすことにより精度が上がるかどうかを検証していない. そのため, どの程度までデータ数を増やすことで精度が向上するか更なる調査が必要である. 

\subsection{抽出のクラスタリングの関係性}
提案手法では自動抽出したオブジェクトを意味に応じてクラスタリングしている. したがって, クラスタリングの精度は自動抽出の精度に依存する. 
RQ1の結果から自動抽出には改善の余地が見られるため, クラスタリングした際に精度の悪さが蓄積する危険性がある. クラスタリングの精度向上には自動抽出の精度向上が必要であることがわかる. 
\chapter{結論}
\label{chap:keturon}

 本研究ではレビューやツイートに含まれるキーフレーズの自動抽出および粒度の細かいクラスタリング手法を提案する. そして, マイニングした結果をWebブラウザ上で表示することで開発者を支援する可視化ツールを提案する. 
具体的には, レビューやツイートの中にあるアプリの欠陥やアプリに対する要望に関するキーフレーズを自動抽出してからその抽出したキーフレーズを意味に応じてクラスタリングする.  そして, マイニングして得られた結果を時系列やアプリごとにWebブラウザ上に出力する.
Google PlayストアだけでなくTwitterのデータも取得, 分析することによりアプリに対するユーザの幅広い情報を活用することができる. また, 抽出したキーフレーズを活用することで機能や内容などに応じた粒度の細かいクラスタリングをすることができる. 最後に得られた結果をグラフなどを用いてWebブラウザ上に表示することにより開発者がレビューを分析する効率を上げることができる. 

3つのRQsを通して, 本研究で作成した自動抽出モデルはレビューの中にキーフレーズがあるかどうかを判断する精度は高いものの, レビューから正確にキーフレーズを抽出する精度に関しては改善の余地が見られた. また, 他の手法と比較することで本研究のクラスタリング手法の精度の高さが確認できた. そして, 開発者にとって本研究で実装した可視化ツールが有用であることを示すことができた. 

今後の課題として, 次を挙げる. 

\begin{itemize}
    \item 対象アプリを増やすことで提案手法の一般性の高さを確認する
    \item App Storeに投稿されるレビューを分析対象に含めることでデータ数を増やす
    \item モデルのパラメータやデータセットの数を変更したときに抽出の精度がどのように変化するのかを検証する
    \item トレーニングデータの精度を改善することで抽出精度を向上させる. 自動抽出の精度を向上させることでクラスタリングの精度向上につながる
    \item Webブラウザへの可視化方法を工夫し, 開発者がよりレビューを分析しやすいようにデザインの改善や機能の拡張を行う
\end{itemize}

%%%%%%%%%%%%%%%%%%%%%%%%%%%%%%%%%%%%%%%%%%%%%%%%%%%%%%%%%%%%%%%%%%
% 論文本体おわり                                                 %
%%%%%%%%%%%%%%%%%%%%%%%%%%%%%%%%%%%%%%%%%%%%%%%%%%%%%%%%%%%%%%%%%%

%%%%%%%%%%%%%%%%%%%%%%%%%%%%%%%%%%%%%%%%%%%%%%%%%%%%%%%%%%%%%%%%%%
% 謝辞. 感謝の気持ちを書く. \acknowledge以下に直接書くか,        %
%   \input で\acknowledge を記述したファイルを読み込む           %
%%%%%%%%%%%%%%%%%%%%%%%%%%%%%%%%%%%%%%%%%%%%%%%%%%%%%%%%%%%%%%%%%%
\acknowledge


本研究を行うにあたり, 日頃から親身にご指導をしていただいた慶應義塾大学理工学部情報工学科の高田眞吾教授に感謝いたします.
そして, 未熟な私を様々な面で身近で支えていただいた慶應義塾大学理工学部情報工学科高田研究室の先輩方, 同期の皆様に心から感謝いたします. 


\begin{flushright}
  {\Large 佐藤 響}
\end{flushright}
%%%%%%%%%%%%%%%%%%%%%%%%%%%%%%%%%%%%%%%%%%%%%%%%%%%%%%%%%%%%%%%%%%
% 謝辞おわり                                                     %
%%%%%%%%%%%%%%%%%%%%%%%%%%%%%%%%%%%%%%%%%%%%%%%%%%%%%%%%%%%%%%%%%%

%%%%%%%%%%%%%%%%%%%%%%%%%%%%%%%%%%%%%%%%%%%%%%%%%%%%%%%%%%%%%%%%%%
% 参考文献                                                       %
% \bibliographystyle に選択した参考文献表示方法を入れる          %
% 参考文献内容は, bibtexで管理するのが楽. thesis.bibに書こう.    %
%%%%%%%%%%%%%%%%%%%%%%%%%%%%%%%%%%%%%%%%%%%%%%%%%%%%%%%%%%%%%%%%%%
\bibliographystyle{jplain}
\bibliography{thesis}
%%%%%%%%%%%%%%%%%%%%%%%%%%%%%%%%%%%%%%%%%%%%%%%%%%%%%%%%%%%%%%%%%%
% 参考文献おわり                                                 %
%%%%%%%%%%%%%%%%%%%%%%%%%%%%%%%%%%%%%%%%%%%%%%%%%%%%%%%%%%%%%%%%%%

% \input{appendix.tex}


\end{document}

