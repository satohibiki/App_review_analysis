\chapter{テンプレートの使用方法}

この章では, 卒論テンプレートの構成と使用方法について述べる.

\section{設定方法}
thesis.tex を自分のパラメータに変更する.

\section{テンプレートのタイプセット}
この卒論テンプレートから卒論をタイプセットするにあたり, 以下の3つが利用可能である.
\begin{itemize}
\item make
\item omake
\item latexmk
\end{itemize}
以下でそれぞれのコマンドでのタイプセット方法について説明する.

\subsection{make}
% によるタイプセットは以下の利点, 欠点がある.
% \begin{description}
% \item [利点:] 非常に簡単にタイプセットを行える.
% \item [欠点:] makeをインストールしなければならない.
% \end{description}

makeでタイプセットを行う方法をコード\ref{code:make}で示す.
このように端末に打ち込むことで, 卒論本体である thesis.pdf が生成される.
また, 生成された中間ファイルを消去する場合は, コード\ref{code:make-clean}を利用すればよい.

\begin{minipage}{\textwidth}
\begin{lstlisting}[caption={makeコマンド}, label={code:make}, language={sh}]
make
\end{lstlisting}

\begin{lstlisting}[caption={中間ファイルの除去}, label={code:make-clean}, language={sh}]
make clean
\end{lstlisting}

\lstinputlisting[caption={Makefile}, label={code:Makefile}, language={make}]{Makefile}
\end{minipage}

\subsection{omake}
後で書く

\subsection{latexmk}
latexmkはTeXLiveに標準で入っている自動タイプセットツールである.
このツールを利用するには, コード\ref{code:latexmk}を実行すればよい. 
また, latexmkの特徴としてファイルの更新の監視を行い, 更新時に自動でタイプセットを行うことが出来る.
この機能を使用するためには, コード\ref{code:latexmk-pvc}を利用すればよい.

\begin{minipage}{\textwidth}
\begin{lstlisting}[caption={latexmkコマンド}, label={code:latexmk}, language={sh}]
latexmk thesis.tex
\end{lstlisting}

\begin{lstlisting}[caption={latexmkによる自動タイプセット}, label={code:latexmk-pvc}, language={sh}]
latexmk -pvc thesis.tex
\end{lstlisting}

\lstinputlisting[caption={.latexmkrc}, label={code:latexmkrc}, language={perl}]{.latexmkrc}
\end{minipage}