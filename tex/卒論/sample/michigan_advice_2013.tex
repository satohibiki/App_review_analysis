\chapter{高田先生からの注意事項(2013年度版)}
以下に述べる事項は, 2013年12月に高田先生が送った
\begin{center}
\fbox{「(重要) 年内の卒論・修論提出について」}
\end{center}
に記載されている内容で, いわゆる卒論ドラフト添削の際の注意事項です.

ここに書いてあるコメントを度々しないといけないようなことはないようにしましょう.

\section{高田先生のチェックについて}
卒論ドラフトはきちんとしたものでないとチェックしません.
つまり,「とりあえず書いた」というものはチェックしません.
提出前に,学生の間で回覧してちゃんとお互いのドラフトをチェックしてください.
1ページ目に誰がチェックしたか書いておいてください.
\begin{center}
\fbox{チェックしたと思えないものは途中で私のチェックはやめます.}
\end{center}

もちろん卒論・修論は,実装が終わっていなくても半分くらい(?)
書けます.しかし,卒論・修論がいっぱい書いてあるのに,実装が
全然終わっていない場合もチェックしません.

\section{卒論ドラフトを添削する際の事前チェック事項}

\begin{minipage}{\hsize}
\begin{enumerate}
\item きちんと推敲してください.
\item 言葉の定義が不十分のまま使用しないこと.
\item 主語,目的語,述語の関係に注意してください.\\
      特に主語と述語がねじれていると本当にわかりにくいです.\\
      私がよく使うルールは,「1文は2文節以内にする」というものです.\\
      (一つの文に文節が二つ以上ありますと,主語と述語の関係が変になりやすい.)
\item 節の最初は,基本的に,省略や代名詞や接続詞を用いないこと.
\item 半角の「(」や「[」の前は,空白を入れること.\\
      × ABC(DEF) \\
      ○ ABC (DEF)
\item 逆茂木文に注意.(逆茂木文が何か分からない学生はヤバいよ.)
\item 卒論・修論にはページ数制限がないので,itemizeやenumerate,
      図などを取り入れて,「わかりやすく」すること.
\item itemizeやenumerateする場合は,その直前に何を列挙しているのかを明示する.
\item 図や表を入れた場合,本文中に必ず説明すること.参照して終わりでは駄目です.
\item 「,.」「、。」の統一
\end{enumerate}
\end{minipage}

\section{論文の構成についてのチェック事項}
論文は基本的に他人に見てもらうような文章です.なので,相手が読みたくなるような書き方をしなければなりません.
「読みたくなるような」文章はどういうものかというのは難しいが,「見た目」は 重要だと思います.
例えば,次のことには注意した方がよいです.

\subsection{論文の見た目について}
\begin{minipage}{\hsize}
\begin{enumerate}
\setlength{\itemsep}{0.8\parindent}
\item 20行以上もある段落は,おそらく構成がなっていません.\\
      どこかで段落わけができることでしょう.
\item 5行以上もある文は,読みにくくなっていることが多いです.\\
      特に逆茂木文になっていないか注意しなければなりません.\\
      (注:「逆茂木文」が何かわからない学生は高田研失格です.)
\item 卒論(および修論)は,何ページ以内とか何word以内という制限はありません.\\
      なので,itemizeやenumerate環境はどんどんつかうべきです.\\
      例えば,「XXX, YYY, ZZZなどの」のような表現でXXX, YYY, ZZZ自身が長いのであれば,itemizeを使うべきです.
\item 図や表も効果的に使うべきです.
\item フォントもある程度使い分けるべきです.
\end{enumerate}
\end{minipage}

\subsection{文章はtop-downに分かるように}
文章はtop-downに書くべきです.つまり,結果を先に書き,そのあとにその結果に関する説明を書くべきです.
そうしないと,文章がダラダラした感じを受けます.

\subsection{展開が論理的か}
話の展開が論理的に進んでいるか注意すべきです.「A→B,よってC」ではなくて,\\「A→B,B→C,よってC」です.

\subsection{単調な部分がないか}
冗長な部分がないか注意してください.\\
第1章にほとんどの人は「論文の構成」を書いています.それを書くのは構いませんが,例えば,次のような書き方をしている人が多いです.\\
    第X章  論文の書き方の基本\\
        論文の書き方の基本について述べる.\\
  これではまるで意味がありません.(当たり前でしょう.)\\
  なので,結論の章を除き,各章に関する説明が2,3文あるようにした
  方がよいです.
  
\subsection{定義があるか}
言葉の定義がきちんと行われているかに注意するべきです.定義がない状態でしばらく使った後に,その言葉を定義する,
というような例がいくつかありました.

\subsection{提案と実装について}
みなさんは「提案」の章と「実装」の章に分けて書いています.
この分け方そのものはもちろんよいのですが,「提案」の章に
あげている項目が「実装」の章のどこで扱われているのか,
一見してわかりにくい卒論・修論が多かったです.実装が提案
したことに対するものなので,対応関係がわかりやすいように
工夫する必要があります.

\subsection{実装は自分の実装だけ}
また,実装の中に長々と過去のシステムの詳細を述べているものが
あります.基本的に切り分けられる場合,他の人がやったことは
別の章に書くべきで,実装の章については自分がやったことを
中心に書くべきです.もし構成上どうしても実装の章に他の人が
やったことを書く場合,最低でもどこが自分でどこが他人が書いた
ものなのか明示しなければなりません.

\subsection{数式は独立させる}
式は一つの行に独立して見やすくするようにしてください.
本文中に埋め込むと,ほとんどの場合,わかりにくいです.

\subsection{論文の長さについて}
長さは50ページ以上が一つの目安ですが,卒論は長ければよいと
いうわけではありません.

\subsection{参照しよう}
「背景」に相当するところでいっぱい書いていても,何を参考にして書いたのかわからない論文は駄目です.
参考文献は明記してください.例えば,章(または節)の頭に,「本章(節)は,××および××を参考にまとめた.」
と書くだけでもいいと思います.

%\section{参考文献の書式について}
%\begin{minipage}{\hsize}
%\begin{itemize}
%\item 第1著者のlast nameのABC順で並べること.
%\item 形式はいろいろとあるが,基本的には:\\
%\fbox{\parbox[c][4zw][l]{37zw}{
%    [Takada 95] Shingo Takada and Norihisa Doi: \\
%    ``An Extended Centering Mechanism for Interpreting Pronouns and Zero-Pronouns'',\\
%     IEICE Transactions on Information and Systems, Vol.E78-D, No.1, pp.58-67 (1995).}}
%  のような形が個人的には好きです.
%\item この形式にこだわる必要はありませんが,少なくとも各自の参考文献リストの中では形式を必ず*** 統一する ***
%\item 会議の場合,場所は省略してもよい.また,月も省略してもよい.\\
%      しかし,ページ番号は必ず記入すること.
%\item ちなみに,[Takada 95]のように [名前 年] の形がなぜ好きかというと[1]のように通し番号よりは情報があるからです.
%\end{itemize}
%\end{minipage}