\chapter{よく使うLaTex記法}
よく使われる図や表の挿入方法について記載する.

\section{図の挿入}
図はfigure環境を使って挿入する.
jpg, pngの挿入も可能だが,pdfで挿入するのがモダンらしい.
captionに図のタイトル,labelに本文中から参照するための識別子を付けることができる.
図\ref{fig:takada}は$\backslash$ref\{fig:takada\}のようにして参照することができる.
figure環境のオプションhtpbの意味や図の大きさ,細かい配置方法については各自調べてみて欲しい.

\begin{minipage}{\textwidth}
  \lstinputlisting[caption={図の挿入}, label={code:figure}, language={tex}]{sample/snippets/figure.tex}
\end{minipage}

\begin{figure}[htpb]
  \centering
  \includegraphics[width=0.4\linewidth]{sample/snippets/takada.jpg}
  \caption{サンプル画像}
  \label{fig:takada}
\end{figure}



\section{表の挿入}
表はtable環境を使って挿入する.
縦の罫線はtabularで|を使って指定する.
横の継戦は$\backslash$hlineを使って指定する.

\begin{minipage}{\textwidth}
  \lstinputlisting[caption={表の挿入}, label={code:table}, language={tex}]{sample/snippets/table.tex}
\end{minipage}

\begin{table}[htpb]
  \centering
  \caption{表の挿入}
  \label{tab:table}
  \begin{tabular}{|l|r|c|} \hline
    左詰 & 右詰 & 中央揃え \\ \hline
    要素は & \&で & 区切る \\ \hline
  \end{tabular}
\end{table}



\section{コードの挿入}
コードの挿入はtexファイルに直接記述することができる.
もちろん,ファイルを指定して挿入することもできる.
詳しい利用法はsample/usage.texを参照.
