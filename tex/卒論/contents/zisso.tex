\chapter{実装}
\label{chap:zisso}

%ーーーーーーーーーーーーーーーーーーーーーーーーーーーー

\section{実装の概要}
本研究では, 膨大の数あるGooglePlayとXに投稿されたアプリレビューをスクレイピングして取得し, レビュー文に含まれるバグレポートやアプリに対する要望を示す箇所を自動抽出, その結果を利用しクラスタリング, 最後にwebブラウザ上に可視化する手法を提案する. 
\begin{description}
\item[実装環境]\mbox{}
\begin{itemize}
 \item オペレーティングシステム
    \begin{itemize}
      \item Mac OS Ventura 13.4.1
    \end{itemize}
 \item 実装言語
    \begin{itemize}
      \item Python 3.11.6
    \end{itemize}
\end{itemize}
\end{description}


\begin{figure}[hbtp]
 \centering
 \includegraphics[width=\linewidth]
      {contents/images/zisso_nagare.png}
 \caption{実装した提案手法の流れ\label{chap:nagare}}
\end{figure}



%ーーーーーーーーーーーーーーーーーーーーーーーーーーーー

\section{対象アプリとレビュー}
本研究では川面による先行研究\cite{kawatsura}のデータセットを使用するため対象アプリは先行研究のアプリに合わせている. 
今回使用するレビューのデータセットは先行研究によって作成された13個のアプリレビューのデータセットに加え, 本研究で新たに収集したアプリレビューのデータセットを使用する. 対象アプリを先行研究に合わせた理由としては, 現在, Xのポスト取得数に制限がある影響で十分な数のツイートが用意できなかったことが原因である. Xのポスト取得数の制限に関しては4.4で詳しく述べる. 
対象となっているアプリを表\ref{tb:taisyouapuri}に示す. 
\begin{table}[htbp]
  \caption{本研究の対象アプリ一覧}
  \label{tb:taisyouapuri}
  \begin{center}
  \begin{tabularx}{\linewidth}{X|l|X}
    \hline
    \mbox{アプリ名}\mbox{(一部略称)}&\mbox{Google Playストアの}\mbox{パッケージID}&\mbox{Twitterの}\mbox{検索キーワード}\\\hline\hline
    にゃんトーク&com.akvelon.meowtalk&にゃんトーク\\\hline
    スマートニュース&jp.gocro.smartnews.android&スマートニュース\\\hline
    PayPay&jp.ne.paypay.android.app&paypay\\\hline
    Coke ON&com.coke.cokeon&coke on\\\hline
    Google Fit&com.google.android.apps.fitness&google fit\\\hline
    Simeji&com.adamrocker.android.input.simeji&simeji\\\hline
    Lemon8&com.bd.nproject&lemon8\\\hline
    楽天ペイ&jp.co.rakuten.pay&楽天ペイ\\\hline
    majica&com.donki.majica&majica\\\hline
    LINE MUSIC&jp.linecorp.linemusic.android&line music\\\hline
    BuzzVideo&com.ss.android.article.topbuzzvideo&buzzvideo\\\hline
    ファミペイ&jp.co.family.familymart\verb|_|app&ファミペイ\\\hline
    CapCut&com.lemon.lvoverseas&capcut\\\hline
  \end{tabularx}\end{center}
\end{table}

%ーーーーーーーーーーーーーーーーーーーーーーーーーーーー

\section{Google Playストアのスクレイピング}
本研究で取得するレビュー情報は先行研究に合わせて下記とする. 

\begin{itemize}
 \item reviewId : レビューID
 \item userName : ユーザ名
 \item userImage : ユーザのプロフィール画像
 \item at : 投稿日時
 \item score : 星の数
 \item content : レビュー内容
 \item thumbsUpCount : このレビューが参考になったと評価した人の数
 \item reviewCreatedVersion : レビュー時のバージョン
 \item replyContent : 開発者からの返信の内容
 \item repliedAt : 開発者からの返信日時
\end{itemize}

先行研究では投稿日時が2021年10月21日〜2021年12月15日までの8週間のレビューを収集している. 用意されたGoogle Playストアの各アプリのレビュー数は表\ref{tb:rawreviewnum}の通りである. 
\begin{table}[htbp]
  \caption{収集したGoogle Playストアのレビュー数(\cite{kawatsura} p.16, 表 4.2)}
  \label{tb:rawreviewnum}
  \begin{center}
  \begin{tabular}{l|l}
    \hline
    アプリ名&収集したレビュー数(件)\\\hline\hline
    にゃんトーク&171\\\hline
    スマートニュース&1,651\\\hline
    PayPay&1,052\\\hline
    Coke ON&1,736\\\hline
    Google Fit&372\\\hline
    Simeji&468\\\hline
    Lemon8&72\\\hline
    楽天ペイ&480\\\hline
    majica&706\\\hline
    LINE MUSIC&359\\\hline
    BuzzVideo&375\\\hline
    ファミマのアプリ&290\\\hline
    CapCut&180\\\hline\hline
    合計&7,912
  \end{tabular}\end{center}
\end{table}
この先行研究のデータに加え, 本研究では2023年10月1日~12月15日のレビューを新たに取得する. 新たに取得されたGoogle Playストアの各アプリのレビュー数は表\ref{tb:rawreviewnum2023}の通りである. 
\begin{table}[htbp]
  \caption{収集したGoogle Playストアのレビュー数(2023/10/1〜12/15)}
  \label{tb:rawreviewnum2023}
  \begin{center}
  \begin{tabular}{l|l}
    \hline
    アプリ名&収集したレビュー数(件)\\\hline\hline
    にゃんトーク&\\\hline
    スマートニュース&\\\hline
    PayPay&\\\hline
    Coke ON&\\\hline
    Google Fit&\\\hline
    Simeji&\\\hline
    Lemon8&\\\hline
    楽天ペイ&\\\hline
    majica&\\\hline
    LINE MUSIC&\\\hline
    BuzzVideo&\\\hline
    ファミマのアプリ&\\\hline
    CapCut&\\\hline\hline
    合計&
  \end{tabular}\end{center}
\end{table}

Google PlayストアのレビューをスクレイピングするためにPythonのプログラムである(\verb|get_google_play_review.py|)を作成した. このプログラムの作成にあたり, Pythonのライブラリであるgoogle-play-scraperを使用する. google-play-scraperでは外部依存関係なしでPython用のGoogle Playストアを簡単にクロールするためのAPIが提供されている\cite{google-play-scraper}. 
このライブラリを使用することにより, アプリのパッケージ名, 言語, 取得する数, 順序を指定してレビューの一覧を取得することができる. 

%ーーーーーーーーーーーーーーーーーーーーーーーーーーーー

\section{Xのスクレイピング}
本研究で取得するツイート情報は先行研究に合わせて下記とする. 
\begin{itemize}
 \item id : ツイートID
 \item content : ツイート内容
 \item at : ツイート日時
\end{itemize}

まず, 先行研究で収集したツイート数を表\ref{tb:rawtweetnum}に示す. 

\begin{table}[htbp]
  \caption{収集したTwitterのツイート数(\cite{kawatsura} p.18, 表 4.3)}
  \label{tb:rawtweetnum}
  \begin{center}
  \begin{tabular}{l|l}
    \hline
    アプリ名&収集したツイート数(件)\\\hline\hline
    にゃんトーク&2,525\\\hline
    スマートニュース&50,590\\\hline
    PayPay&880,319\\\hline
    Coke ON&84,424\\\hline
    Google Fit&13,496\\\hline
    Simeji&205,327\\\hline
    Lemon8&4,376\\\hline
    楽天ペイ&11,111\\\hline
    majica&3,649\\\hline
    LINE MUSIC&184,873\\\hline
    BuzzVideo&41,656\\\hline
    ファミマのアプリ&8,867\\\hline
    CapCut&33,998\\\hline\hline
    合計&1,525,211
  \end{tabular}\end{center}
\end{table}

この先行研究に加え, 本研究では新たに2023年10月1日~12月15日のポストを取得する. Xのポスト取得に関してはTwitter APIを使用してスクレイピングを行う. Twitter APIのプランに関してはFree, Basic, Pro, Enterpriseの4つのプランが用意されておりそれぞれ料金や使用できる機能などが異なる. 大規模なサービスやビジネス向けのEnterpriseプラン以外の3つのプランの違いの一部を表\ref{tb:xplan}に示す. 
表\ref{tb:xplan}よりポストを取得するためにはBasicプラン以上に加入する必要がある. Basicプランに加入した場合でも合計で30,000件しか取得されないため本研究では先行研究のデータセットを追加で使用することとした. Xの利用規約によると, Xが提供するインターフェイスを介して行うスクレイピング以外は禁止としている. そのため, seleniumなどを使用したスクレイピングは断念し, 過去の論文のデータを使用することとした. 


\begin{table}[htbp]
  \caption{プランとできること}
  \label{tb:xplan}
  \begin{center}
  \begin{tabular}{|l|c|c|c|}
    \hline
    &Free&Basic&Pro \\\hline\hline
    料金&無料&月額100ドル&月額5,000ドル \\\hline
    月間ポスト数の上限&1,500&3,000&300,000 \\\hline
    月間ポスト取得数&0&10,000&1,000,000 \\\hline
  \end{tabular}\end{center}
\end{table}

Twitter APIを使用してポストを取得するためにPythonのプログラムである(\verb|get_tweet.py|)を作成した. このプログラムではTwitter APIにアクセスするためのライブラリであるTweepy\cite{tweepy}を使用した. まずAPIキーなどの4つの認証情報をセットする. 次にClientクラスのsearch\_resent\_tweetメソッドを使用してツイートを取得する. 
このメソッドは最大過去7日間まで遡ってツイートを取得できる. search\_all\_tweetsメソッドでは全てのツイートを取得できるが, ``Academic Research''という学術用の用途でAPI承認されたユーザーしか使用できないため本研究では使用しなかった. 
先行研究と同じ情報を取得するために, 本研究では引数として以下のものを与えた. 
\begin{itemize}
 \item max\_resul : 検索結果の最大数. 10〜100の数値で, デフォルトは10
 \item query: 検索ワード
 \item tweet\_field: ツイートフィールドを選択. 今回はツイート日時を取得するために["created\_at"]とした. 
 \item end\_time: 期間の終わりを指定できる(UTCタイムスタンプ)
\end{itemize}

新たに取得したXのポスト数を表\ref{tb:rawtweetnum2023}に示す

\begin{table}[htbp]
  \caption{収集したXのポスト数}
  \label{tb:rawtweetnum2023}
  \begin{center}
  \begin{tabular}{l|l}
    \hline
    アプリ名&収集したツイート数(件)\\\hline\hline
    にゃんトーク&\\\hline
    スマートニュース&\\\hline
    PayPay&\\\hline
    Coke ON&\\\hline
    Google Fit&\\\hline
    Simeji&\\\hline
    Lemon8&\\\hline
    楽天ペイ&\\\hline
    majica&\\\hline
    LINE MUSIC&\\\hline
    BuzzVideo&\\\hline
    ファミマのアプリ&\\\hline
    CapCut&\\\hline\hline
    合計&
  \end{tabular}\end{center}
\end{table}

%ーーーーーーーーーーーーーーーーーーーーーーーーーーーー

\section{前処理}
機械学習によるレビューに含まれる有用な箇所の自動抽出の精度を上げるために, GooglePlayストアとXから取得したデータに対して前処理を行うプログラム(\verb|preprocessing_google.py|, \verb|preprocessing_twitter.py|)を作成した. この処理では以下に示す処理を行う. この処理は一般的な自然言語処理の手法を参考としている. 
\begin{itemize}
  \item 英語を全て小文字に揃える. 
  \item 以下の文字列を削除. 
    \begin{itemize}
      \item 「」【】()()『』
      \item @@から始まるメンション
      \item \#から始まるタグ
      \item URL
      \item 半角空白,全角空白
      \item 絵文字
      \item 日本語を含まないレビュー
    \end{itemize}
  \item レビューやツイートには, 異なるバグの報告や新しい機能の要望に関する文が2文以上からなるものがある. そのため, 「。」「.」「!」「!」「?」「!」「\verb|\n|」「\verb|\r\n|」でそれぞれの文に分割する. 
\end{itemize}
以下の図\ref{chap:preprocessing}がレビューを前処理した例である. 2つの文で構成されているため「。」で区切り分割している. また絵文字は削除されている. 

\begin{figure}[hbtp]
 \centering
 \includegraphics[scale=0.5]
      {contents/images/preprocessing.png}
 \caption{前処理の例\label{chap:preprocessing}}
\end{figure}

前処理した結果をcsvファイルにて保存する. 保存する項目としては, 投稿日時(at), レビューのid(reviewId)またはツイートid(id), そして, 前処理した文章である. 図\ref{tb:googlecsv}, 図\ref{tb:twittercsv}に前処理結果後のcsvファイルの一部を示す. 

\begin{table}[htbp]
  \caption{Google Playストアレビューの前処理結果(buzzvideo)}
  \label{tb:googlecsv}
  \begin{center}
  \begin{tabularx}{\linewidth}{|l|l|X|}
    \hline
    at&reviewId&content\\\hline\hline
    2021-12-15 19:25:30&gp:AOqpTOHj6w ...&バズビデオを見て、感動をありがとう\\\hline
    2021-12-15 12:28:09&gp:AOqpTOHleV ...&内容が残酷で異常な人が多い\\\hline
    2021-12-15 11:09:50&gp:AOqpTOHG7O ...&分かりづらい\\\hline
    2021-12-14 15:16:33&gp:AOqpTOGWvT ...&ばず29さいって人が投稿してる動画すべて虚偽動画なのでアカウント削除と動画削除して欲しい\\\hline
    2021-12-14 15:16:33&gp:AOqpTOGWvT ...&あるだけで大迷惑です\\\hline
    2021-12-14 15:16:33&gp:AOqpTOGWvT ...&二度と登録し直せないよう個体識別番号で縛ってください\\\hline
    2021-12-14 15:16:33&gp:AOqpTOGWvT ...&お願いします\\\hline
  \end{tabularx}\end{center}
\end{table}

\begin{table}[htbp]
  \caption{ツイートの前処理結果(BuzzVideo)}
  \label{tb:twittercsv}
  \begin{center}
  \begin{tabularx}{\linewidth}{|l|l|X|}
    \hline
    at&id&content\\\hline\hline
    2021-12-15T23:55:11.000Z&1471267626655825922&芸能人に似てる気がするけど名前が思い出せない\\\hline
    2021-12-15T23:53:43.000Z&1471267256659509249&驚愕男性が豆乳を飲むべき3つの理由\\\hline
    2021-12-15T23:53:43.000Z&1471267256659509249&男だからこそ注目したい豆乳のメリットとは\\\hline
    2021-12-15T23:53:17.000Z&1471267149746679813&感情を乗せた歌声と歌詞に聞き惚れちゃう♪壊れかけのradio\\\hline
    2021-12-15T23:53:10.000Z&1471267120264904705&kkと眞子の酷い嘘\\\hline
    2021-12-15T23:53:10.000Z&1471267120264904705&恐ろしい真実が明らかに\\\hline
  \end{tabularx}\end{center}
\end{table}

%ーーーーーーーーーーーーーーーーーーーーーーーーーーーー

\section{有用な箇所の自動抽出}
\subsection{概要}
膨大な数あるレビュー文から開発に有用なレビューを絞り込み, かつ有用なレビュー文の中からバグの報告やアプリに対する要望に関して記述している部分を自動抽出する. 自動抽出のために日本語のデータで事前学習済みの言語表現モデルである日本語BERTに対して質問応答形式のfune-tuningを行うことで自動抽出器を生成する. 
抽出を行う文章の特徴をモデルに理解させるために質問文にその文章がGooglePlayストアのレビューなのかXの文章なのかという「カテゴリー」の情報と「アプリ名」という2つの情報を加える. 図\ref{chap:fine-tuning}に質問応答形式によるfine-tuningのモデルを示す. 
\begin{figure}[hbtp]
  \centering
  \includegraphics[scale=0.3]
       {contents/images/fine-tuning.png}
  \caption{前処理の例\label{chap:fine-tuning}}
 \end{figure}

また, 図\ref{chap:answer}に質問文とその答えの例を示す. 質問文にアプリの欠陥やアプリに対する要望を尋ねる文章を与え, その答えとして欠陥や要望を示す箇所を返すようにしている. 
\begin{figure}[hbtp]
  \centering
  \includegraphics[scale=0.4]
       {contents/images/answer.png}
  \caption{前処理の例\label{chap:answer}}
 \end{figure}
\subsection{データセット}
データセットとして, Google PlayストアとXのツイートからそれぞれ5,000件ずつ合計で10,000件のデータをランダムに抽出し手作業で有用な箇所を抽出した. データセットの作成は情報工学科の学部4年生2人がそれぞれ手作業で行い, お互いの抽出した箇所が異なっていたものは議論することにより決定した. 
データセットには前処理したデータを利用したcsvファイルを使用する. このcsvファイルにはid,アプリ名,投稿日時,本文, 手動で抽出した結果の5つの情報が入っている. idは前処理したデータを識別するために与えられ, Google Playストアのレビュー文はg\_{index}, twitterのツイートのidはt\_{index}とする. 
10,000件のデータセットのうち, 6,000件を訓練データ, 2,000件を検証用データ, 2,000件をテストデータとする. 質問応答形式のfine-tuningを行うために, csv形式であるデータセットをソースコード\ref{json}に示すようにjson形式に変換する. 

\begin{lstlisting}[caption=データセット.json,label=json]
  {
    "version": "v2.0", 
    "data": [
      {
        "title": "モバイルアプリのレビュー", 
        "paragraphs": [
          {
            "context": "本アカウントのフォローやリツイートお願いします",
            "qas": [
              {
                "id": "t_2223388",
                "question": "この文はTwitterのツイートです。
                             paypayアプリの欠陥やpaypayアプリに対する
                             要望が書かれているのはどこですか?",
                "is_impossible": true,
                "plausible_answers": [{"text": "", "answer_start": -1}],
                "answers": [{"text": "", "answer_start": -1}]
              }
            ]
          },
          {
            "context": "11/25前後からアプリを開いても強制終了、
                      会員バーコードもクーポンも何も出せない状態、
                      これでは買い物ができないと、こちらのレビュー
                      を見に来て沢山の方が同じ状態であることが
                      わかった",
            "qas": [
              {
                "id": "g_6041", 
                "question": "この文章はGooglePlayストアのレビューです。
                            majicaアプリの欠陥やmajicaアプリに対する
                            要望が書かれているのはどこですか?",
                "answers": [{"text": "アプリを開いても強制終了、
                                      会員バーコードもクーポン
                                      も何も出せない", 
                             "answer_start": -1}], 
                "is_impossible": false
              }
            ]
          }, ...
        ]
      }
    ]
  } 
\end{lstlisting}

このjsonファイルは下記の構成になっている. 
\begin{itemize}
  \item version: バージョンを表す. 今回は答えられない質問を含むSQuAD 2.0と同じバージョンのため, v2.0とする
  \item title: contextのタイトル
  \item paragraphs: context1つとそれに関連する質問, 答えがリスト形式で保持されている
  \item qas: 質問と回答がリスト形式となっている
  \item context: 元の文章(抽出する前の文章)
  \item id: 設定したid
  \item question: 質問文
  \item is\_impossible: 答えられない質問ならtrue, それ以外はfalse
  \item plausible\_answers: 質問が答えられない時のみ存在し, 問題文から答えになりうる部分を抽出
  \item answers: contextから抜き出した答えとその位置情報がリスト形式で保持されている. 答えを複数用意することもできる. 
  \item text: contextから抜き出した答えのテキスト情報(抽出する文章)
  \item answe\_start: contextから抜き出した答えの位置情報
\end{itemize}

\subsection{モデルのfine-tuning}
用意したデータセットを用いてモデルをfine-tuningする. 本研究では事前学習済みモデルを提供するフレームワークであるHuging FaceのTransformersを通して利用できる東北大学のモデル\cite{tohoku}を使用する. このモデルは日本語のWikipediaのデータを用いて学習されている\cite{tohoku}. 
この東北大学が公開している日本語BERTのうち, whole word maskingを適用して学習させているモデル\cite{masking}を用いる. whole word maskingとは事前学習時に単語ごとでマスクするかどうかを決め, マスクする単語に対応するサブワードを全てマスクする方式である. モデルのパラメータは以下に示す通りである. 
\begin{itemize}
  \item 学習率: 3e-5
  \item エポック数: 10
  \item バッチサイズ: 12
\end{itemize}

実装にはTransformersに含まれるスクリプトであるrun\_squad.pyを用いる. 

\subsection{自動抽出}
fine-tuningを行ったモデルを使用して自動抽出を行う. 前処理したGooglePlayストアのデータ14,051件と, twitterのデータ4,634,319件のデータから有用な箇所を自動抽出する. これにより開発に有用なレビューのみを選択でき, その文章の中に含まれるバグの報告やアプリに対する要望に関して記述している部分を抽出できる. 

結果は表\ref{tb:googleqa}に示すようにcsv形式で保存する. 

\begin{table}[htbp]
  \caption{Google Playストアレビューの自動抽出結果(google\_fit)}
  \label{tb:googleqa}
  \small
  \begin{center}
  \begin{tabularx}{\linewidth}{|l|l|X|X|X|}
    \hline
    id&app\_name&datetime&context&prediction\\\hline\hline
    g\_955&coke\_on&2021-11-27 11:17:03&商品が出ない事が何回か発生しました&商品が出ない\\\hline
    g\_956&coke\_on&2021-11-02 12:15:37&使用している端末が、利用できる端末の一覧表にないため、サポートは期待できない&使用している端末が、利用できる端末の一覧表にない\\\hline
    g\_959&coke\_on&2021-11-11 15:32:50&そもそも自販機側が黄色点滅していなくて買えないことが多過ぎです&自販機側が黄色点滅していなくて買えない\\\hline
    g\_961&coke\_on&2021-11-14 23:13:26&今まではcoke\_on対応を優先してかっていたが、これからはコカコーラ製品全般をできるだけ買わないようにする&今まではcoke\_on対応を優先してかっていたが、これからはコカコーラ製品全般をできるだけ買わないようにする\\\hline
    g\_964&coke\_on&2021-10-24 12:30:07&自販機との接続を早くしてほしい&自販機との接続を早くしてほしい\\\hline
    g\_965&coke\_on&2021-11-11 16:43:39&やっと繋がっても先にキャンペーン広告が出てすぐに買えないのが不親切&キャンペーン広告が出てすぐに買えない\\\hline
    g\_969&coke\_on&2021-12-07 08:28:27&コークオンパスのフリー20プランの残り回数が分かりやすく表示してほしい&コークオンパスのフリー20プランの残り回数が分かりやすく表示してほしい\\\hline
    g\_973&coke\_on&2021-11-23 14:51:45&反応しない&反応しない\\\hline
    g\_977&coke\_on&2021-11-11 18:22:04&2本以上の購入はとても使えません&2本以上の購入はとても使えません\\\hline
    g\_978&coke\_on&2021-11-01 11:06:09&それと対応自販機との連携が悪い&自販機との連携が悪い\\\hline
    g\_980&coke\_on&2021-10-22 03:22:06&動きが遅い&動きが遅い\\\hline
    g\_982&coke\_on&2021-11-22 11:58:45&ただ自分のスマホのストレージが小さく、データ容量が大きいため今回一先ず削除いたします&自分のスマホのストレージが小さく、データ容量が大きい\\\hline
  \end{tabularx}\end{center}
\end{table}
%ーーーーーーーーーーーーーーーーーーーーーーーーーーーー

\section{クラスタリング}
抽出した文章のクラスタリングには既存研究のクラスタリング手法\cite{sira}を参考にして実行する. この手法は以下の3つの手順である. 
\begin{enumerate}
  \item Universal Sentence Encoder (USE)を用いて, 問題のある特徴語句を512次元のベクトルに変換
  \item 重み付き無向グラフを構成し, 各問題機能をノードとし, 2つの問題機能のUSEベクトル間のコサイン類似度スコア(比率)をノード間の重みとする. 比率は入力ハイパーパラメータであり, 問題のある機能間の意味的相関を測定. ハイパーチューニングした閾値は0.5とした
  \item このグラフに対して, Chinese Whispers (CW)を実行し, 問題のある特徴量をクラスタリング
\end{enumerate}

既存研究では英語の文章をベクトルに変換するためにUniversal Sentence Encoder (USE)を使用しているが, 今回は対象が日本語の文章であるため, Sentence-BERTの日本語モデルを使用する. Sentence-BERTとは, 事前学習されたBERTモデルとSiamese Networkを使い, 高品質な文ベクトルを作る手法である. このモデルを使用することで, 高品質な文ベクトルが作成できる. 
したがって本研究でのクラスタリング手法は下記である. 
\begin{enumerate}
  \item Sentence-BERTの日本語モデルを用いて, 抽出した文章をベクトルに変換
  \item 重み付き無向グラフを構成し, 各問題機能をノードとし, 2つの抽出した文章間のベクトル間のコサイン類似度スコア(比率)をノード間の重みとする. 比率は入力ハイパーパラメータであり, 抽出した文章間の意味的相関を測定. 検証した結果, 閾値は0.8とした
  \item このグラフに対して, Chinese Whispers (CW)を実行し, 問題のある特徴量をクラスタリング
\end{enumerate}

これにより, 抽出した文章をその文章が示す意味に応じてクラスタリングすることができる. それぞれの文章にクラスタの番号(以下 : クラスタ番号)が振られ, クラスタ番号が同じものが同じクラスタとなり番号が近いものは意味的相関が近いことを表す. 

結果はcsvファイルに保存される. 表\ref{tb:clustering}に示されるように抽出した文章にクラスタ番号が振られる. 

\begin{table}[htbp]
  \caption{抽出した文章とクラスタ番号(google\_fit)}
  \label{tb:clustering}
  \begin{center}
  \begin{tabularx}{\linewidth}{|X|c|}
    \hline
    prediction&cluster\\\hline\hline
    十分歩いて108歩とかふざけんな&273\\\hline
    278歩に減っていた&274\\\hline
    再起動しても直らない&275\\\hline
    接続/連携を適宜確認しておく必要がある&276\\\hline
    歩いた歩数より足りない&279\\\hline
    使えない&280\\\hline
    使えない&280\\\hline
    歩けない&280\\\hline
    使えない&280\\\hline
    動かなかった&280\\\hline
    反応しない&280\\\hline
    使い方も分からない&280\\\hline
    使えない&280\\\hline
    使えない&280\\\hline
    動かなくなった&280\\\hline
    長期放置されてるんでしょうか&281\\\hline
    下がるって何故でしょうか&282\\\hline
    カウントされなくなる&283\\\hline
    何もカウントしなくなった&283\\\hline
    カウント出来ていない&283\\\hline
    カウントされず&283\\\hline
    記録ができませんと&283\\\hline
    計測しなくなった&283\\\hline
    カウントされなくなった&283\\\hline
    カウントしない&283\\\hline
    全くカウントされていない&283\\\hline
    カウントしなくなりました&283\\\hline
    データが反映されなくなった&283\\\hline
  \end{tabularx}\end{center}
\end{table}


%ーーーーーーーーーーーーーーーーーーーーーーーーーーーー

\section{画面出力・可視化}
クラスタリングした結果をwebブラウザ上で可視化する. webアプリケーションの実装に使用した言語, フレームワークは以下となっている. 
\begin{itemize}
    \item フロントエンド: HTML/CSS, JavaScript
    \item バックエンド: Python
    \item フレームワーク: Flask
\end{itemize}

アコーディオンメニューを使用して抽出した文章の一覧をクラスタごとに表示する. また, 抽出した文章をクリックすると投稿日時や元のレビュー文がモーダルウィンドウで表示する. 
さらにレビューが投稿された期間やレビュー文のキーワードで絞り込み検索することが可能となっている. そして, 日ごとのレビュー数を表す折れ線グラフとクラスタに含まれるレビュー数の上位10個を表す棒グラフを作成した. この2つのグラフは検索結果に応じて動的に変化するようになっている. 