\chapter{結論}
\label{chap:keturon}

レビューの中にある問題のある機能やアプリに対する要望に関する報告を自動抽出してからその抽出した情報を元に分類することにより分類精度を向上させつつ, 粒度の高い分類を行う. また, その結果を時系列やアプリごとにWebブラウザ上に出力し, 開発者を支援する可視化ツールを提案した.

3つのRQにて自動抽出, 分類の性能を評価し, 可視化の有用性を示すことができた. 